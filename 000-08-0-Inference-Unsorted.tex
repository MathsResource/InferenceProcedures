\subsection{F-test of equality of variances}
The test statistic is

\begin{equation} F = \frac{S_X^2}{S_Y^2}\end{equation}

has an F-distribution with $n-1$ and $m-1$ degrees of freedom if the null hypothesis of equality of variances is true.


\subsubsection{The F-test}
$H_0$: Both variances are equal
$H_a$ : The variances are different.
Compute the test statistic.
Divide the larger variance by the smaller variance.
The degrees of freedom are as follows

$\nu_1$ size of sample with larger variance
$\nu_2$ size of sample with smaller variance

There are 5 values tabulated
We use the one for a significance level of 0.05
Carefully read the tables.

\section{Inference}

We will use a simplistic system for interpreting significance values (i.e. p-values).

If a p value is less than 0.02 we reject the nyll hypothesis.
If the p value is greater than 0.05 we fail to reject the null hypothesis
If the pvalue us between the two thresholds then we deem the procedure to be inconclusive. 

The null hypothesis is that the true correlation coefficient is zero (which is to say, no linear relationship exists). 




\section{Interpreting p-values}

The p-value is the probability of having observed our data (or more extreme data) when the null hypothesis is true 

The smaller the p-value, the less likely it is that the sample results come from a situation where the null hypothesis H0 is true. If the p-value is sufficiently small, we reject the null hypothesis, and support the alternative hypothesis Ha.

\begin{framed}	
	\textbf{One Sided Tests}
	\begin{itemize}
		\item 		p-value  >  0.05   :   no evidence against H0 in favour of Ha
		
		\item 	p-value    <  0.05   :   evidence against H0 in favour of Ha
	\end{itemize}	
	\textbf{Two Sided Tests}
	\begin{itemize}
		\item 	p-value    >  0.025   :   no evidence against H0 in favour of Ha
		
		\item 	p-value    <  0.025   :   evidence against H0 in favour of Ha
	\end{itemize}		
\end{framed}





\subsection{Hypothesis Testing : Two Populations}

Two samples drawn from two populations are independent samples if
the selection of the sample from population 1 does not affect the
selection of the sample from population 2. The following notation
will be used for the sample and population measurements:

\begin{itemize}
	\item $p_1$ and $p_2$ = means of populations 1 and 2,
	
	\item $\sigma_1$ and $\sigma_2$ = standard deviations of
	populations 1 and 2,
	
	\item $n_l$ and $n_2$ = sizes of the samples drawn from
	populations 1 and 2 ($n_1 >30 $, $n_2 >30 $),
	
	\item $x_1$ and $x_2$, = means of the samples selected from
	populations 1 and 2,
	
	\item $s_{1}$ and $s_{2}$ = standard deviations of the samples
	selected from populations 1 and 2.
	
\end{itemize}

	


\section{Statement of the Null and Alternative Hypotheses}
\[H_0 : \pi_1 = \pi_2\]
\[H_1 : \pi_1 \neq \pi_2\]

\[H_0 : \pi_1 - \pi_2 = 0\]
\[H_1 : \pi_1 -  \pi_2 \neq 0\]

Expected Value of differences under null hypothesis

$\pi_1 - \pi_2 = 0$


Significance level = 0.01

\[SE(p_1 - p_2) = \sqrt{\bar{p}(1-\bar{p})\left( \frac{1}{n_1} + \frac{1}{n_2} \right)  }\]

Calculate Pooled Proportion Estimate

\[ \bar{p} = \frac{29 + 62}{1110 + 1553} \]

Test Statistic

\[ \frac{(p_1 - p_2) - (\pi_1 - \pi_2)}{SE(\pi_1 - \pi_2)} \]


\begin{itemize}
	\item Standard Error = 0.007123
	\item Test Statistic = -1.965
\end{itemize}


