\documentclass[]{article}
\voffset=-1.0cm
\oddsidemargin=0.0cm
\textwidth = 480pt
\usepackage[utf8]{inputenc}
\usepackage[english]{babel}
\usepackage{framed}
\usepackage{graphicx}
\usepackage{enumerate}% http://ctan.org/pkg/enumerate
\usepackage{multicol}
\usepackage{amsmath}
\usepackage{amssymb}
\usepackage[angle=0,scale=1,color=black,hshift=-0.4cm,vshift=15cm]{background}
\usepackage{multirow}
\usepackage{eurosym}
\usepackage{vmargin}
\usepackage{amsmath}
\usepackage{graphics}
\usepackage{epsfig}
\usepackage{subfigure}
\usepackage{fancyhdr}
\usepackage{listings}
\usepackage{framed}



\begin{document}




\subsection{Null Hypothesis}
\begin{itemize}
	\item The null hypothesis, H0, represents a theory that has been put forward, either because it is believed to be true or because it is to be used as a basis for argument, but has not been proved. For example, in a clinical trial of a new drug, the null hypothesis might be that the new drug is no better, on average, than the current drug. We would write H0: there is no difference between the two drugs on average.
	
	\item  We give special consideration to the null hypothesis. This is due to the fact that the null hypothesis relates to the statement being tested, whereas the alternative hypothesis relates to the statement to be accepted if / when the null is rejected.
	
	\item  The final conclusion once the test has been carried out is always given in terms of the null hypothesis. We either "Reject H0 in favour of H1" or "Do not reject H0"; we never conclude "Reject H1", or even "Accept H1".
	
	\item  If we conclude "Do not reject H0", this does not necessarily mean that the null hypothesis is true, it only suggests that there is not sufficient evidence against H0 in favour of H1. Rejecting the null hypothesis then, suggests that the alternative hypothesis may be true.
	
\end{itemize}

\subsection{Alternative Hypothesis}
The alternative hypothesis, H1, is a statement of what a statistical hypothesis test is set up to establish. For example, in a clinical trial of a new drug, the alternative hypothesis might be that the new drug has a different effect, on average, compared to that of the current drug. We would write

\begin{description}
	\item[H1:] the two drugs have different effects, on average.
\end{description}


The alternative hypothesis stands in opposition to the Null Hypothesis. In this case, a suitable null hypothesis might be:

\begin{description}
	\item[H0:] the two drugs have the same effect, on average.
\end{description}
A different choice of alternative hypothesis might be that the new drug is better, on average, than the current drug. In this case we would write H1: the new drug is better than the current drug, on average. 

The final conclusion once the test has been carried out is always given in terms of the null hypothesis. We either "Reject H0 in favour of H1" or "\textit{Do not reject H0}". We never conclude "\textit{Reject H1}", or even "\textit{Accept H1}".

\subsection{Hypothesis Testing is not Proof}
If we conclude "Do not reject H0", this does not necessarily mean that the null hypothesis is true, it only suggests that there is not sufficient evidence against H0 in favour of H1. Rejecting the null hypothesis then, suggests that the alternative hypothesis may be true.





\subsection{Hypothesis tests (Null and Alternative Hypotheses) }

%The phrase "test of significance" was coined by Ronald Fisher;
%"Critical tests of this kind may be called tests of significance, and when such tests are available we may discover whether a second sample is or is not significantly different from the first." \\
\begin{itemize}
	\item The null hypothesis (which we will denoted $H_0$) is an hypothesis about a population parameter, such as the population mean $\mu$. \item The purpose of hypothesis testing is to test the viability of the null hypothesis in the light of experimental data. \item The alternative hypothesis $H_1$ expresses the exact opposite of the null hypothesis. \item Depending on the data, the null hypothesis either will or will not be rejected as a viable possibility in favour of the alternative hypothesis.
\end{itemize}



\subsection{The Null Hypothesis }

\begin{itemize}
	\item The null hypothesis is what the experimenter supposes the outcome before the test is performed, based on prior assumptions (note:  future remarks on the Dice experiment will be based on this view).
	\item An alternative view is that the null hypothesis is often the reverse of what the experimenter actually believes; it is put forward to allow the data to contradict it. \item In a hypothetical experiment on the effect of sleep deprivation, the experimenter probably expects sleep deprivation to have a harmful effect. \item If the experimental data show a sufficiently large effect of sleep deprivation, then the null hypothesis ,expressing that sleep deprivation has no effect, can be rejected.
\end{itemize}



%--------------------------------------------------------------------------------------------------------------------------%

\subsection{The Null Hypothesis }

\begin{itemize}
	\item Hypothesis tests are almost always performed using null-hypothesis tests.
	
	\item The rationale is as follows: ``Assuming that the null hypothesis is true, what is the probability of observing a value for the test statistic that is at least as extreme as the value that was actually observed?"
	\item
	The critical region of a hypothesis test is the set of all outcomes which, if they occur, will lead us to decide that there is a difference.
	\item That is, cause the null hypothesis to be rejected in favour of the alternative hypothesis.
	% \item (Remark: Selecting a suitable critical region is arbitrary (for later) ).
\end{itemize}





\subsection{The alternative Hypothesis}

\begin{itemize}
	\item The alternative hypothesis is a statement that contradists the null hypothesis about the value of a population parameter.
	\item The alternative hypothesis is denoted $H_1$ or $H_a$
	\item It must contain a condition of equality. (i.e. ` = ' , `$ \leq$' or `$\geq$')
\end{itemize}

\subsection{The alternative Hypothesis}

\begin{itemize}
	\item Suppose we have a die. We do not know if it is a fair die or a crooked die.
	\item If it was a fair die, the expected value of each throw is 3.5.
	\item If it is a crooked die, the expected value is not 3.5.
	\item To be notationally consistent with the rest of the material in this section, we will denoted the expected value as $\mu$ rather $E(x)$.
\end{itemize}

$H_0 : \mu  = 3.5$  The die is fair.
$H_1 : \mu  \neq 3.5$ The die is crooked.

If you are conducting a study and want to use a hypothesis to suppprt your claim, the claim must be worded so that it becomes the alternative hypothesis.
The null hypothesis is written as a statement that is a direct contradition of this.

%----------------------------------------------------------------------------------------------------%

\subsection{Number of Tails (For Later) }

\begin{itemize}
	\item The alternative hypothesis indicates the number of tails.
	\item A rule of thumb is to consider how many alternative to the $H_0$ is offered by $H_1$.
	\item When $H_1$ includes either of these relational operators;`$>$' ,`$<$' , only one alternative is offered.
	\item When $H_1$ includes the $\neq$ relational operators, two alternatives are offered (i.e.`$>$' or `$<$').
\end{itemize}


\subsubsection{Two tailed test}

$H_0:  = $
$H_1:  \neq $

$\alpha$ is divided equally between the two tail of the critical region.

$\neq$ i.e. \emph{``not equal to"} can also mean \emph{``less than or greater than"}.





\end{document}
