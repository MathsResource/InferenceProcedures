
\documentclass[a4]{beamer}
\usepackage{amssymb}
\usepackage{graphicx}
\usepackage{subfigure}
\usepackage{newlfont}
\usepackage{amsmath,amsthm,amsfonts}
%\usepackage{beamerthemesplit}
\usepackage{pgf,pgfarrows,pgfnodes,pgfautomata,pgfheaps,pgfshade}
\usepackage{mathptmx} % Font Family
\usepackage{helvet} % Font Family
\usepackage{color}
\mode<presentation> {
\usetheme{Default} % was Frankfurt
\useinnertheme{rounded}
\useoutertheme{infolines}
\usefonttheme{serif}
%\usecolortheme{wolverine}
% \usecolortheme{rose}
\usefonttheme{structurebold}
}
\setbeamercovered{dynamic}
\title[MA4704]{Technology Maths 4 \\ {\normalsize Lecture 11A}}
\author[Kevin O'Brien]{Kevin O'Brien \\ {\scriptsize kevin.obrien@ul.ie}}
\date{Spring 2013}
\institute[Maths \& Stats]{Dept. of Mathematics \& Statistics, \\ University \textit{of} Limerick}
\renewcommand{\arraystretch}{1.5}
%----------------------------------------------------------------------------------------------------------%
\begin{document}

\begin{frame}
\titlepage
\end{frame}



\begin{frame}
\frametitle{This Lecture}
This Lecture will look at hypothesis tests and confidence intervals for proportions, and concludes the `Inferences Procedures' section of the course (corresponding to Question 4 of the end of year paper). In part, it acts as a revision for this part of the course also.
\begin{itemize}
\item Some Remarks about one tailed tests
\item Confidence Interval For Proportions
\item Hypothesis test for a proportion (single sample)
\item Hypothesis test for difference in proportions from two populations
\item Estimation an appropriate for a proportion estimate
\end{itemize}
\end{frame}

%----------------------------------------------------------------------------------------------------------%
\begin{frame}
\frametitle{One Tailed Hypothesis test}
\begin{itemize}
\item A one-sided test is a statistical hypothesis test in which the values for which we can reject the null hypothesis, $H_0$ are located entirely in one tail of the probability distribution.

\item In other words, the critical region for a one-sided test is the set of values less than the critical value of the test, or the set of values greater than the critical value of the test.

\item A one-sided test is also referred to as a one-tailed test of significance.

\item A rule of thumb is to consider the alternative hypothesis.  If only one alternative is offered by $H_1$ (i.e. a $<$ or a $>$ is present, then it is a one tailed test.)
\item (When computing quantiles from Murdoch Barnes table 7, we set $k=1$)
\item For the sake of brevity, we have focussed two-tailed procedures in this module, as those procedures are far more commonly used. Awareness of one-tailed procedures is encouraged.
\end{itemize}
\end{frame}

\end{document} 