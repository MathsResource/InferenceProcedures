\documentclass[a4paper,12pt]{article}
%%%%%%%%%%%%%%%%%%%%%%%%%%%%%%%%%%%%%%%%%%%%%%%%%%%%%%%%%%%%%%%%%%%%%%%%%%%%%%%%%%%%%%%%%%%%%%%%%%%%%%%%%%%%%%%%%%%%%%%%%%%%%%%%%%%%%%%%%%%%%%%%%%%%%%%%%%%%%%%%%%%%%%%%%%%%%%%%%%%%%%%%%%%%%%%%%%%%%%%%%%%%%%%%%%%%%%%%%%%%%%%%%%%%%%%%%%%%%%%%%%%%%%%%%%%%
\usepackage{eurosym}
\usepackage{vmargin}
\usepackage{amsmath}
\usepackage{graphics}
\usepackage{epsfig}
\usepackage{subfigure}
\usepackage{framed}
\usepackage{enumerate}
\usepackage{fancyhdr}

\setcounter{MaxMatrixCols}{10}
%TCIDATA{OutputFilter=LATEX.DLL}
%TCIDATA{Version=5.00.0.2570}
%TCIDATA{<META NAME="SaveForMode"CONTENT="1">}
%TCIDATA{LastRevised=Wednesday, February 23, 201113:24:34}
%TCIDATA{<META NAME="GraphicsSave" CONTENT="32">}
%TCIDATA{Language=American English}

\pagestyle{fancy}
\setmarginsrb{20mm}{0mm}{20mm}{25mm}{12mm}{11mm}{0mm}{11mm}
\lhead{MS4222} \rhead{Kevin O'Brien} \chead{Confidence Interval} %\input{tcilatex}

\begin{document}

\section*{Confidence Interval - Worked Example}

In a study of 400 households in the suburbs of a large city, 240 responded
that they have digital television.
\\
\bigskip
\textbf{Question:} Compute the 95\% confidence interval for the proportion of all suburban residents that have digital television.


\begin{framed}
General Structure of Confidence Intervals

\[
\mbox{Point Estimate} \; \pm \; \left( \mbox{Quantile} \times \mbox{Standard Error} \right)
\]
\end{framed}



The \textbf{point estimate} is the sample proportion.
\begin{itemize}
	\item $n$ is the sample size. \\
	For our example $n = 400$.
	\vspace{0.1cm}
	\item $\hat{P}$ is the sample proportion
	\[ \hat{P} = \frac{240}{400} = 0.60 \]
	( 60 \% expressed as percentage)
	\vspace{0.1cm}
	\item $1 - \hat{P}$ is the complement of the sample proportion.\\ For our example $1- \hat{P} = 1 - 0.60  =  0.40.$
	
\end{itemize}


\textbf{Quantile}
\begin{itemize}
	\item Large sample ($n >30$).
	\item We use the standard normal ($Z$) distribution.
	\item The quantile depends on the required confidence level
\end{itemize}
\begin{center}
	\begin{tabular}{|c|c|}
		\hline \phantom{sp} Confidence Level \phantom{sp}  &  \phantom{sp} Quantile \phantom{sp} \\ 
		\hline  95\% & 1.96  \\ 
		\hline  99\% & 2.576 \\ 
		\hline  99.9\%& 3.29  \\ 
		\hline 
	\end{tabular} 
\end{center}

%%- \frametitle{Confidence Intervals for Proportions}

\noindent The standard error, used to compute confidence interval for proportions, is computed according to the following formula.



\[
S.E. (\hat{p}) = \sqrt{ { \hat{p} \times ( 1 - \hat{p}) \over n}}
\]

\noindent In terms of percentages, we would use this variant.
\[
S.E. (\hat{p}) = \sqrt{ { \hat{p} \times ( 100 - \hat{p}) \over n}}
\]


\begin{itemize}
	\item This standard error formula is different from the one from hypothesis testing.
	\item The relevant formulae for standard errors are generally included at the back of exam papers.
\end{itemize}




\noindent Using these values, we can calculate the standard error with this expression.


\[
S.E. (\hat{P}) = \sqrt{ { 0.60 \times 0.40 \over 400}}
\]



%%- \frametitle{Confidence Intervals for Proportions}

However, it is often easier to perform such calculations when working in terms of percentages.
\vspace{0.1cm}
\[
S.E. (\hat{P}) = \sqrt{ { \hat{P} \times ( 100 - \hat{P}) \over n}}
\;[\%] \]

\[
S.E. (\hat{P}) = \sqrt{ { 60 \times 40 \over 400}}  \;[\%]
\]

\[
S.E. (\hat{P}) = \sqrt{ 6 } = 2.45  \;[\%]
\]



%%- \frametitle{Confidence Intervals for Proportions}

\noindent	\textbf{Putting It Together}
\begin{itemize}
	\item Point Estimate:  $\hat{p} = 60\% $
	\vspace{0.2cm}
	\item Quantile: $Z = 1.96 $
	\vspace{0.2cm}
	\item Standard Error:
	$S.E. (\hat{P}) = 2.45 \% $
\end{itemize}
\vspace{0.5cm}

\noindent Confidence Interval (in terms of percentages)
\[  (55.2\%, 64.8 \%)\]

\noindent Confidence Interval (in terms of proportions)
\[  (0.552, 0.648)\]




\end{document}
