\documentclass[]{report}

\voffset=-1.5cm
\oddsidemargin=0.0cm
\textwidth = 480pt

\usepackage{framed}
\usepackage{subfiles}
\usepackage{graphics}
\usepackage{newlfont}
\usepackage{eurosym}
\usepackage{amsmath,amsthm,amsfonts}
\usepackage{amsmath}
\usepackage{color}
\usepackage{amssymb}
\usepackage{multicol}
\usepackage[dvipsnames]{xcolor}
\usepackage{graphicx}
\begin{document}

%%\chapter{12. Inference Procedures for Single Samples }


%--------------------------------------------------------%

\section{Difference in proportions}
We can also construct a confidence interval for the difference between two sample proportions, $\pi_1 - \pi_2$. The point estimate is the difference in sample proportions for tho both groups , $\hat{p}_1- \hat{p}_2$.\\\bigskip


\subsubsection{Estimation Requirements}
The approach described in this lesson is valid whenever the following conditions are met:

\begin{itemize}
\item Both samples are simple random samples.
\item The samples are independent.
\item Each sample includes at least 10 successes and 10 failures.
\item The samples comprises less than 10\% of their respective populations.
\end{itemize}




\subsubsection{Confidence Interval}
\textbf{Compute the standard Error}

\[ S.E. (\hat{p}_1 - \hat{p}_2) =
\sqrt{ \left[{\hat{p}_1 \times (1 - \hat{p}_1) \over n_1}\right] + \left[{\hat{p}_2 \times (1 - \hat{p}_2) \over n_2}\right] } \]

\[ S.E. (\hat{p}_1 - \hat{p}_2) =
\sqrt{ \left[{40 \times 60 \over 400}\right] + \left[{30 \times 70 \over 300}\right] }  = \sqrt{ \left[{2400 \over 400}\right] + \left[{2100\over 300}\right] } \]

\[ S.E. (\hat{p}_1 - \hat{p}_2)
= \sqrt{ 6 + 7 } = 3.6\% \]








\section{Standard Error for Difference of Proportions}

\[ S.E. (\hat{p}_1 - \hat{p}_2) =
\sqrt{ \left[{\hat{p}_1 \times (1 - \hat{p}_1) \over n_1}\right] + \left[{\hat{p}_2 \times (1 - \hat{p}_2) \over n_2}\right] } \]
\begin{itemize}
\item $\hat{p}_1$ and $\hat{p}_2$ are the sample proportions of groups 1 and 2 respectively.
\item $n_1$ and $n_2$ are the sample sizes of groups 1 and 2 respectively.
\end{itemize}
N.B. This formula will be provided in the exam paper. Also, there is no accounting for small samples.




\section*{Confidence Interval for the Difference Between Two Proportions}
\begin{itemize}
\item A confidence interval gives us some idea of the range of values which an unknown population parameter (such as the mean or variance) is likely to take based on a given set of sample data.

\item Many occasions arise where we have to compare the proportions of two different populations. 
\item For example, a firm may want to compare the proportions of defective items produced by different machines; medical researchers may want to compare the proportions of men and women who suffer heart attacks etc. 
\item A confidence interval for the difference between two proportions would specify a range of values within which the difference between the two true population proportions may lie, for such examples.

\item The procedure for obtaining such an interval is based on the sample proportions, p1 and p2, from their respective overall populations.
\end{itemize}



%--------------------------------------------------------%

\section{Testing the Difference Between Two Population Proportions}
\begin{itemize}
\item When we wish to test the hypothesis that the proportions in two populations are not different, the two sample proportions are pooled as a basis for determining the standard error of the
difference between proportions.

\item Note that this differs from the procedure used previously on statistical estimation, in which
the assumption of no difference was not made.

\item Further, the present procedure is conceptually similar to that presented in Section 11.1, in which the two sample variances are pooled as the basis for computing the standard error of the difference between means.
\end{itemize}




\subsection{Hypothesis Tests of Differences between Proportions }


This procedure is used to compare two proportions from two different populations. For two tailed tests, the null hypothesis states that the population proportion $\pi_1-\pi_2$ has a specified value, with the alternative stating that  $\pi_1-\pi_2$ does not have this value. 



\begin{framed}
\noindent \textbf{Specifying the Null and Alternative Hypothesis}\\

\begin{multicols}{2}
\[H_0 : \pi_1 = \pi_2\]
\[H_1 : \pi_1 \neq \pi_2\]

\[H_0 : \pi_1 - \pi_2 = 0\]
\[H_1 : \pi_1 -  \pi_2 \neq 0\]

\end{multicols}
\end{framed}


\begin{itemize}
\item Expected Value of differences under null hypothesis: $\pi_1 - \pi_2 = 0$

\item Significance level = 0.01

\[SE(p_1 - p_2) = \sqrt{\bar{p}(1-\bar{p})\left( \frac{1}{n_1} + \frac{1}{n_2} \right)  }\]

\item Calculate Pooled Proportion Estimate

\[ \bar{p} = \frac{29 + 62}{1110 + 1553} \]

\item Test Statistic

\[ \frac{(p_1 - p_2) - (\pi_1 - \pi_2)}{SE(\pi_1 - \pi_2)} \]

\end{itemize}
%\begin{multicols}{2}
%\begin{itemize}
%\item Standard Error = 0.007123
%\item Test Statistic = -1.965
%\end{itemize}
%
%\end{multicols}


%-------------------------------------------------------------------------------------------%
\begin{framed}
The formula for the estimated standard error is:

\[ S.E (\hat{p}_1 - \hat{p}_2)  = \sqrt{\bar{p}(100- \bar{p}) \left( {1 \over n_1} + {1 \over n_2}  \right)} \]


where $\bar{p}$ is a aggregate proportion (proportion of successes from overall sample, regardless of which group they are in).
\end{framed}







\subsection{Pooled Estimate for Population Proportion}
The pooled estimate of the population proportion, based on the proportions obtained in two independent samples.









\section{Worked Example}




\textbf{Proportions : Remarks}
\begin{itemize}
\item Small Sample sizes will not be considered for the case of sample proportions or Difference of proportions. Small samples will considered for ``sample mean" cases only.
\end{itemize}


\begin{itemize}
\item The computed p-values is compared to the pre-specified significance level of 5\%. Since the p-value ($<0.0001$) is less than the significance level of 0.05, the effect is statistically significant. 

\item Since the effect is significant, the null hypothesis is rejected. The conclusion is that the probability of graduating from high school is greater for students who have participated in the early childhood intervention program than for students who have not. 

\item The results could be described in a report as:
The proportion of students from the early-intervention group who graduated from high school was 0.86 whereas the proportion from the control group who graduated was only 0.52. The difference in proportions is significant, with $p < 0.0001$.
\end{itemize}




\section{Summary of Inference Procedures}
%------------------------------------%
\textbf{Point Estimates:}
\[\hat{p}_1 = \frac{x_1}{n_1}  \]
\[\hat{p}_2 = \frac{x_2}{n_2}  \]

%------------------------------------%
\texttt{Hypotheses:}
\[H_0 : \qquad \pi_1 \leq \pi_2   \]
\[H_1 : \qquad \pi_1 > \pi_2   \]

\begin{itemize}
\item The population proportion for group 1 does not exceed the corresponding value for group 2. 
\item The population proportion for group 1 does exceed (is greater than) the corresponding value for group 2.
\end{itemize}

\[H_0 : \qquad \pi_1 - \pi_2  \leq 0  \]
\[H_1 : \qquad \pi_1 - \pi_2 > 0   \]

%------------------------------------%

\textbf{Critical Vale}
\begin{itemize}
\item $\alpha = 0.05$
\item One-tailed Procedure (refer back to $H_1$) k=1
\item Large sample ($x_1+x_2 > 30$)
\end{itemize}

\textbf{Descision}
is $|TS| > CV$?

Comclusion: We can reject the null hypothesis, We can reasonably conclude that....



\newpage



%--------------------------------------------------------%

\textbf{Difference in proportions}
This lesson describes how to construct a confidence interval for the difference between two sample proportions, p1 - p2.
\textbf{Estimation Requirements}
The approach described in this lesson is valid whenever the following conditions are met:

\begin{itemize}
	\item Both samples are simple random samples.
	\item The samples are independent.
	\item Each sample includes at least 10 successes and 10 failures.
	\item The samples comprises less than 10\% of their respective populations.
\end{itemize}


%--------------------------------------------------------%


\textbf{Standard Error for Difference of Proportions}

\[S.E. (\hat{P}_1 - \hat{P}_2)  = \sqrt{ [P_1 \times (1 - P_1) / n_1] + [P_2 \times (1 - P_2) / n_2] } \] 
\begin{itemize}
	\item $\hat{P}_1$ and $\hat{P}_2$ are the sample proportions of groups 1 and 2 respectively.
	\item $n_1$ and $n_2$ are the sample sizes of groups 1 and 2 respectively.
\end{itemize}
N.B. This formula will be provided in the exam paper.


%--------------------------------------------------------%


\begin{itemize}
	\item SE = $\sqrt{ [p_1 \times (1 - p_1) / n_1] + [p_2 \times (1 - p_2) / n_2] } $
	\item SE = $\sqrt{ [0.40 \times 0.60 / 400] + [0.30 \times 0.70 / 300] } $
	\item SE  = $\sqrt{[ (0.24 / 400) + (0.21 / 300) ]}$ = $\sqrt{(0.0006 + 0.0007)}$ = sqrt(0.0013) = 0.036 
\end{itemize}





\end{document}
