\documentclass[]{report}

\voffset=-1.5cm
\oddsidemargin=0.0cm
\textwidth = 480pt

\usepackage{framed}
\usepackage{subfiles}
\usepackage{graphics}
\usepackage{newlfont}
\usepackage{eurosym}
\usepackage{amsmath,amsthm,amsfonts}
\usepackage{amsmath}
\usepackage{color}
\usepackage{amssymb}
\usepackage{multicol}
\usepackage[dvipsnames]{xcolor}
\usepackage{graphicx}
\begin{document}
%-------------------------------------------------%


\section{Part 2: One Sample proportion test}
\[ H_0: \pi = 60\%\]
\[ H_1: \pi \neq 60\%\]














%------------------------------------------------------------------------------%
{
\textbf{Computing the Standard Error}

\[
S.E. (\hat{p}) \;=\; \sqrt{ {72 \times 28 \over 200 }}
\]


}

%------------------------------------------------------------------------------%
{

\textbf{Computing the Standard Error}

\[
S.E. (\hat{p}) \;=\; \sqrt{ {\hat(p) \times (100 -\hat{p} )\over n}}
\]



\[
\hat{p} = {144/200}  \times 100\%  = 0.72 \times 100\%.  = 72%
\]

$100\% - \hat{p} = 100\% - 72\% = 28\% $


\textbf{Computing the Standard Error}

\[
S.E. (\hat{p}) \;=\; \sqrt{ {72 \times 28 \over 200 }}
\]


}


%-----------------------------------------------------------%


\subsubsection{CI for Proportion}

\begin{itemize}
\item $\hat{p}  = 0.62$
\item Sample Size $n=250$
\item Confidence level $1-\alpha$ is $95\%$
\end{itemize}




%-----------------------------------------------------------%

\textbf{CI for Proportion: Example (2)}

\begin{itemize}
\item First, lets determine the quantile.
\item The sample size is large, so we will use the Z distribution.
\item (Alternatively we can uses the $t-$ distribution with $\infty$ degrees of freedom.
\end{itemize}

%-----------------------------------------------------------%

\subsubsection{CI for Proportion}

\begin{itemize}
\item First, lets determine the quantile.
\item The sample size is large, so we will use the Z distribution.
\item (Alternatively we can uses the $t-$ distribution with $\infty$ degrees of freedom.
\end{itemize}





\section{Computing the Standard Error}

\[
S.E. (\hat{p}) \;=\; \sqrt{ {\hat(p) \times (100 -\hat{p} )\over n}}
\]




\[
\hat{p} = {144/200}  \times 100\%  = 0.72 \times 100\%.  = 72%
\]

$100\% - \hat{p} = 100\% - 72\% = 28\% $




\[
S.E. (\hat{p}) \;=\; \sqrt{ {72 \times 28 \over 200 }}
\]

\newpage

%------------------------------------------------------------------------------%

%------------------------------------------------------------------------------%
{

\textbf{ Standard Error for Proportions}

The standard error for proportions is computed using this formula.
\[
S.E. (\hat{p}) \;=\; \sqrt{ {\hat{p} \times (1-\hat{p} )\over n}}
\]

\smallskip
When expressing the proportion as a percentage, we adjust the standard error accordingly.
\[
S.E. (\hat{p}) \;=\; \sqrt{ {\hat{p} \times (100 -\hat{p} )\over n}}
\]
}

\textbf{ Sample Proportion : Example}


\begin{description}
\item [Point Estimate] The sample proportion is computed as follows
\[ \hat{p} = \frac{x}{n} = \frac{56}{160} = 0.35\]
\item [Quantile] We are asked for a 95\% confidence interval. The quantile is therefore
\[ z_{\alpha/2} =1.96\]
\item [Standard Error] The standard error, with sample size n=120 is computed as follows
\[ \mbox{S.E.}(\hat{p}) = \sqrt{\frac{\hat{p} \times (1-\hat{p})}{n}} =  \sqrt{\frac{0.35 \times 0.65}{160}}\]

\end{description}
\noindent\textit{(Full solution to follow)}
\newpage


%-----------------------------------------------------------%


\subsection{CI for Proportion: Example (1)}

\begin{itemize}
\item $\hat{p}  = 0.62$
\item Sample Size $n=250$
\item Confidence level $1-\alpha$ is $95\%$
\end{itemize}


%-----------------------------------------------------------%


\begin{itemize}
\item First, lets determine the quantile.
\item The sample size is large, so we will use the Z distribution.
\item (Alternatively we can uses the $t-$ distribution with $\infty$ degrees of freedom.
\end{itemize}




%------------------------------------------------------------------------------%

Although the sample mean is useful as an unbiased estimator of the population mean, there is no way of
expressing the degree of accuracy of a point estimator. In fact, mathematically speaking, the probability that the
sample mean is exactly correct as an estimator of the population mean is $P = 0$.

%------------------------------------------------------------------------------%

A confidence interval for the
mean is an estimate interval constructed with respect to the sample mean by which the likelihood that the interval
includes the value of the population mean can be specified.

The \emph{level of confidence} associated with a confidence interval indicates the long-run percentage
of such intervals which would include the parameter being estimated.

%------------------------------------------------------------------------------%

\begin{itemize}
\item Confidence intervals for the mean typically are constructed with the unbiased estimator $\bar{x}$ at the midpoint
of the interval.

\item The $\pm Z \sigma_x$ or $\pm Z s_x$ frequently is called the \textbf{\emph{margin of error}} for the confidence interval.
\end{itemize}

%------------------------------------------------------------------------------%

We indicated that use of the normal distribution in estimating a population mean is warranted
for any large sample ($n > 30$), \textbf{and} for a small sample ($n \leq 30$) only if the population is normally distributed
and $\sigma$ is known.

%------------------------------------------------------------------------------%

\begin{itemize}
\item Now we consider the situation in which the sample is small and the population is normally distributed,
but $\sigma$ is not known.
\item The distribution is a family of distributions, with
a somewhat different distribution associated with the degrees of freedom ($df$). For a confidence interval for the
population mean based on a sample of size n, $df = n - 1$.
\end{itemize}


%------------------------------------------------------------------------------%
{

\subsection{Computing the Standard Error}

\[
S.E. (\hat{p}) \;=\; \sqrt{ {\hat(p) \times (100 -\hat{p} )\over n}}
\]



}

%
------------------------------------------------------------------------------%

{
\[
\hat{p} = {144/200}  \times 100\%  = 0.72 \times 100\%.  = 72%
\]

$100\% - \hat{p} = 100\% - 72\% = 28\% $

}


%------------------------------------------------------------------------------%
{
\textbf{Computing the Standard Error}

\[
S.E. (\hat{p}) \;=\; \sqrt{ {72 \times 28 \over 200 }}
\]


}
%------------------------------------------------------------------------------%

\newpage

\bigskip
\textbf{Computing the Standard Error}

\[
S.E. (\hat{p}) \;=\; \sqrt{ {35 \times 65 \over 160 }} =  3.77\%
\] \bigskip

\textbf{Confidence Interval for proportion}

\[
35 \pm (1.96 \times 3.77) \%  = (35 \pm7.4) \% = (27.6\%,42.4\%)
\]

\newpage

%-----------------------------------------------------------%

% Confidence Interval for a Proportion
% One Sample
%------------------------------------------------------------------------------%
{
\textbf{Computing the point estimate}

Sample percentage

\[
\hat{p} = \frac{x}{n} \times 100%
\]

\begin{itemize}
\item $\hat{p}$ - sample proportion.
\item $x$  - number of ``successes".
\item $n$  - the sample size.
\end{itemize}

}



%-------------------------------------------------------------------------------%

\begin{framed}
\begin{itemize}
\item \textbf{Standard Error for Hypothesis Testing}

\[\sqrt{\frac{\pi (1- \pi)}{n}}\]

\item \textbf{Standard Error for Confidence Intervals}


\[\sqrt{\frac{\bar{p} (1- \bar{p})}{n}}\]
\end{itemize}
\end{framed}

\newpage



\section{Summary of Inference Procedures}
%------------------------------------%
\textbf{Point Estimates:}
\[\hat{p}_1 = \frac{x_1}{n_1}  \]
\[\hat{p}_2 = \frac{x_2}{n_2}  \]

%------------------------------------%
\texttt{Hypotheses:}
\[H_0 : \qquad \pi_1 \leq \pi_2   \]
\[H_1 : \qquad \pi_1 > \pi_2   \]

\begin{itemize}
\item The population proportion for group 1 does not exceed the corresponding value for group 2. 
\item The population proportion for group 1 does exceed (is greater than) the corresponding value for group 2.
\end{itemize}

\[H_0 : \qquad \pi_1 - \pi_2  \leq 0  \]
\[H_1 : \qquad \pi_1 - \pi_2 > 0   \]
%-------------------------------------%
\textbf{Standard Error}
First we computed the aggregate sample proportion $\bar{p}$.

\[\bar{p} = \frac{x_1 + x_2}{n_1 + n_2}\]
%------------------------------------%

The Standard Error is 
\[ S.E.(\pi_1 - \pi_2) = \sqrt{\bar{p} \times (100-\bar{p}) \times \left(\frac{1}{n_1}+ \frac{1}{n_2} \right)}\]
(\textit{Given in formula sheet})
%------------------------------------%
\textbf{Standard Error}

The Test Statistic is therefore 

\[ TS = \frac{(\hat{p}_1-\hat{p}_2)-(\pi_1 - \pi_2)}{S.E.(\pi_1 - \pi_2)}\]

%------------------------------------%

\textbf{Critical Vale}
\begin{itemize}
\item $\alpha = 0.05$
\item One-tailed Procedure (refer back to $H_1$) k=1
\item Large sample ($x_1+x_2 > 30$)
\end{itemize}

\textbf{Descision}
is $|TS| > CV$?

Comclusion: We can reject the null hypothesis, We can reasonably conclude that....



\end{document}

%-----------------------------------------------------------%



