
\subsection{Point Estimates}

In statistics, estimation refers to the process by which one makes inferences about a population, based on information obtained from a sample.
There are two types of estimations:

\begin{itemize}
	\item Point Estimation
	\item Interval Estimation
\end{itemize}


%%- \frametitle{Point Estimates}

An estimate of a population parameter may be expressed in two ways:

Point estimate. A point estimate of a population parameter is a single value of a statistic. 


% http://www.math.uah.edu/stat/point/


Point estimation refers to the process of estimating a parameter from a probability 
distribution, based on observed data from the distribution. 



% http://www.cliffsnotes.com/math/statistics/principles-of-testing/point-estimates-and-confidence-intervals


You have seen that the sample mean $\bar{x}$ is an unbiased estimate of the population mean $\mu$. 
Another way to say this is that $\bar{x}$  is the best point estimate of the true value of $\mu$. 



Some error is associated with this estimate, however—the true population mean may be larger 
or smaller than the sample mean. Instead of a point estimate, you might want to identify a 
range of possible values $p$ might take, controlling the probability that $\mu$ is not lower 
than the lowest value in this 
range and not higher than the highest value. Such a range is called a confidence interval.



\section{Interval Estimates}

An interval estimate is the range of values used to estimate an unknown parameter together with the probability, known as the \emph{confidence level}, that the unknown population parameters in within that range. Confidence intervals are conventionally centered around the point estimate.

The two numbers defining confidence interval are the lower confidence limit and the upper confidence limit, collectively as the confidence limits. A confidence interval expresses the degree of accuracy or confidence we have in an estimate.



