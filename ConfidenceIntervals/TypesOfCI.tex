	\documentclass[a4paper,12pt]{article}
%%%%%%%%%%%%%%%%%%%%%%%%%%%%%%%%%%%%%%%%%%%%%%%%%%%%%%%%%%%%%%%%%%%%%%%%%%%%%%%%%%%%%%%%%%%%%%%%%%%%%%%%%%%%%%%%%%%%%%%%%%%%%%%%%%%%%%%%%%%%%%%%%%%%%%%%%%%%%%%%%%%%%%%%%%%%%%%%%%%%%%%%%%%%%%%%%%%%%%%%%%%%%%%%%%%%%%%%%%%%%%%%%%%%%%%%%%%%%%%%%%%%%%%%%%%%
\usepackage{eurosym}
\usepackage{vmargin}
\usepackage{framed}
\usepackage{amsmath}
\usepackage{graphics}
\usepackage{epsfig}
\usepackage{subfigure}
\usepackage{enumerate}
\usepackage{fancyhdr}

\setcounter{MaxMatrixCols}{10}
%TCIDATA{OutputFilter=LATEX.DLL}
%TCIDATA{Version=5.00.0.2570}
%TCIDATA{<META NAME="SaveForMode"CONTENT="1">}
%TCIDATA{LastRevised=Wednesday, February 23, 201113:24:34}
%TCIDATA{<META NAME="GraphicsSave" CONTENT="32">}
%TCIDATA{Language=American English}

\pagestyle{fancy}
\setmarginsrb{20mm}{0mm}{20mm}{25mm}{12mm}{11mm}{0mm}{11mm}
\lhead{MS4222} \rhead{Kevin O'Brien} \chead{Confidence Intervals} %\input{tcilatex}

\begin{document}

%--------------------------------------------------%
\section*{Types of Confidence Intervalfor Sample Means}
\begin{framed}
\noindent \textbf{Confidence Intervals for Sample Means}\\
Broadly speaking, there are three different types of confidence interval. \\
\textit{(Type 2 and Type 3 are more important for us)}
\begin{description}
\item[Type 1] Sample with \textbf{known} population variance
\begin{itemize}
\item The size of the sample doesn't matter.
\end{itemize}
\item[Type 2] Large sample with \textbf{known} population variance
\begin{itemize}
\item The size of the sample is more than 30 ($n > 30$)
\end{itemize}
\item[Type 3] Small sample with \textbf{known} population variance
\begin{itemize}
\item The size of the sample is 30 or less ($n\leq 30$)
\end{itemize}
\end{description}
\end{framed}



\subsection*{Type 1 : Known Population Variance}

\begin{itemize}
\item This type of confidence interval is very rare in practice, but it is very simple to implement and used as introductory 
teaching material with those studying confidence intervals for the first time.
\item This type of confidence interval is computed using this formula:
\[ \bar{x} \pm z_{(\alpha/2)}{\sigma \over \sqrt{n}} \]
%\item A description of each item is on the next slide.


\item The point estimate is the sample mean : $\bar{x}$

\item We use a \textbf{\textit{quantile}} from the standard normal (Z) distribution to ``scale"
the confidence interval to the specified confidence level (usually 95\%).

\begin{itemize}
    \item[$\ast$] Let the confidence level be denoted in the for $(1-\alpha)\times 100\%$, and hence determine $\alpha$.
For example, if the confidence level is 95\%, then $\alpha$ is 0.05 (or 5\%).

\item[$\ast$] The Quantile ($z_{(\alpha/2)}$) is the value for the Standard Normal Tables (for example Murdoch Barnes Table 3) such that
\[ P(Z \geq z_{(\alpha/2)}) = {\alpha \over 2}\]

\item[$\ast$] For a 95\% confidence interval, the quantile is 1.96. For a 99\% confidence interval, the quantile is approximately 2.58. 
\end{itemize}


\item The population standard deviation is $\sigma$. The sample size is $n$.
\item The standard error to be used in this confidence interval is
\[ \mbox{S.E}(\bar{x}) = \frac{\sigma}{\sqrt{n}}\]
\end{itemize}





%--------------------------------------------------%

\subsection*{Type 2 : Large Sample, Unknown Population Variance}
\begin{itemize}

\item The point estimate is the sample mean : $\bar{x}$

\item For a type 2 confidence interval, we can determine a \textbf{\textit{quantile}} for the confidence interval in the same way that for the Type 1 confidence interval.

\item Recall: For a 95\% confidence interval, the quantile is 1.96. For a 99\% confidence interval, the quantile is approximately 2.58.


\item The population standard deviation which we denote $\sigma^2$ is unknown.
Instead we are given the sample variance $s^2$. We use the sample standard deviation (the square root of the variance) as an estimate for the population standard deviation $\sigma$.
\[ s \mbox{ is an estimate for } \sigma \]

\item The sample size is $n$.
\item The standard error to be used in a Type 2 confidence interval is
\[ \mbox{S.E}(\bar{x}) = \frac{s}{\sqrt{n}}\]

\end{itemize}





%--------------------------------------------------%

\subsection*{Type 3 : Small Sample, Unknown Population Variance}
\begin{itemize}
\item Firstly, we will clarify the similarities of the Type 2 and Type 3 confidence interval.
\item The point estimate is the sample mean $\bar{x}$.
\item The sample standard deviation $s$ is used to estimate the population $\sigma$.
\item The standard error is
\[ \mbox{S.E}(\bar{x}) = \frac{s}{\sqrt{n}}\]


\item The key difference is in determining the quantile. Rather than use the standard normal distribution, we must use the student $t-$ distribution. 
\item Quantiles for this distribution are also tabulated in statistical tables (for Example, Murdoch Barnes Table 7).
\item Recall that we must determine a value for $\alpha$ ( and hence $\alpha/2$). For a 95\% confidence interval, $\alpha= 0.05$  and $\alpha/2 = 0.025$.
\item Computing a quantile from the $t-$ distribution additionally requires the specification of the \textit{\textbf{degrees of freedom}}. Degrees of Freedom are often denote as $df$ or by the greek letter $\nu$ (``nu").
\item For small sample confidence intervals (i.e. $n \leq 30$), the degrees of freedom are
\[ df = n-1 \]
\end{itemize}














\subsection*{Small samples}
\begin{itemize} \item We indicated that use of the normal distribution in estimating a population mean is warranted
for any large sample ($n > 30$). \item For a small sample ($n \leq 30$) only if the population is normally distributed
\textbf{and} $\sigma$ is known, the standard normal distribution can be used compute quantiles. In practice,
this case is unusual.
\item Now we consider the situation in which the sample is small and the population is normally distributed,
but $\sigma$ is not known.
\end{itemize}




%--------------------------------------------------%

\subsection*{Using the $t-$distribution for large samples}

\begin{itemize}
\item The $t-$distribution is used for computing quantiles in the case of small samples (i.e. when sample size $n \leq 30$).
\item A key value in the $t-$distribution is the degrees of freedom, denoted $df$ (or sometimes $\nu$). For small samples \[ df= n-1\].
\item The $t-$distribution is used for computing quantiles in the case of large samples too, as an alternative to using the $Z$ distribution.
\item In this case , use the value $\infty$ as the degrees of freedom (see bottom row of tables).
\[ df= \infty\]
\item This means that we can use the $t-$ distribution for finding the quantiles of all types of confidence intervals.

\end{itemize}


\subsection*{Confidence Intervals for Sample Proportion}

\begin{itemize}
\item The Structure of a confidence interval for sample proportion is
\[ \hat{p} \pm z_{(\alpha/2)} \times \mbox{S.E.}(\hat{p})\]

\item The standard error, in the case of sample proportions, is
\[ \mbox{S.E.}(\hat{p}) = \sqrt{\frac{\hat{p}\times (1-\hat{p})}{n}}\]

\end{itemize}

\noindent Unlike confidence intervals for sample means, there is only one type of confidence interval when dealing with sample proportions.

\newpage
\noindent \textbf{Optional}
\begin{itemize}

\item It is often easier to work in terms of percentages, rather than proportions.
If you are working in terms of percentages, remember to use the appropriate \textbf{\textit{complement value}} in the standard error formula (i.e. $100 - \hat{p}$). 

\item The standard error, in the case of sample proportions, is
\[ \mbox{S.E.}(\hat{p}) = \sqrt{\frac{\hat{p}\times (100-\hat{p})}{n}} \;\;\;\; [\%]\]
\item For the sake of clarity, the percentage symbol can be included in bracket beside the formula.
\item \textbf{Complement Values:}
\begin{itemize} \item[$\ast$] When working in terms of proportions, for the the value $\hat{p} =0.40$, the complement value is $1-\hat{p} =0.60$.
\item[$\ast$] When working in terms of percentages, for the value $\hat{p} = 40\%$, the complement value is $100-\hat{p} = 60\%$.
\end{itemize}
\end{itemize}


\end{document}

%-----------------------------------------------------------%


