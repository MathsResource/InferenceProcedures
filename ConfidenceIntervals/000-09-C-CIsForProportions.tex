\documentclass[]{report}

\voffset=-1.5cm
\oddsidemargin=0.0cm
\textwidth = 480pt

\usepackage{framed}
\usepackage{subfiles}
\usepackage{graphics}
\usepackage{newlfont}
\usepackage{eurosym}
\usepackage{amsmath,amsthm,amsfonts}
\usepackage{amsmath}
\usepackage{color}
\usepackage{amssymb}
\usepackage{multicol}
\usepackage[dvipsnames]{xcolor}
\usepackage{graphicx}
\begin{document}
%===========================================================================%

\subsection{Confidence Intervals for Sample Proportion}
Unlike confidence intervals for sample means, there is only one type of confidence interval when dealing with sample proportions.

\textbf{Optional}
\begin{itemize}

\item It is often easier to work in terms of percentages, rather than proportions.
If you are working in terms of percentages, remember to use the appropriate \textbf{\textit{complement value}} in the standard error formula (i.e. $100 - \hat{p}$)

\item The standard error, in the case of sample proportions, is
\[ \mbox{S.E.}(\hat{p}) = \sqrt{\frac{\hat{p}\times (100-\hat{p}}{n}}\]

\item Complement Values:
\begin{itemize} \item When working in terms of proportions, for the the value $\hat{p} =0.35$, the complement value is $1-\hat{p} =0.65$.
\item When working in terms of percentages, for the value $\hat{p} = 35\%$, the complement value is $100-\hat{p} = 65\%$.
\end{itemize}
\end{itemize}




\section{Confidence Intervals for Sample Proportion}
Confidence Intervals may also be computed for \textbf{Sample Proportions}. \\The sample proportion is used to estimate the value of a population proportion. The sample proportion is denoted $\hat{p}$. The population proportion is denoted $\pi$.

\begin{framed}
	\textbf{Confidence Intervals for Sample Proportion}
	
	\begin{itemize}
		\item The Structure of a confidence interval for sample proportion is 
		\[ \hat{p} \pm z_{(\alpha/2)} \times \mbox{S.E.}(\hat{p})\]
		
		\item The standard error, in the case of sample proportions, is
		\[ \mbox{S.E.}(\hat{p}) = \sqrt{\frac{hat{p}\times (1-\hat{p})}{n}}\]
		\item (When computing this interval with statistical software, it is common to enhance the solution using a \textbf{continuity correction} . This is not part of our syllabus. )
	\end{itemize}
\end{framed}

%%%%%%%%%%%%%%%%%%%%%%%%%%%%%%%%%%%%%%%%%%%%%%%%%%%%%%%%%%%%%%%%%%%%%%%%%%%%%

\section{Point Estimates for proportions }
\textbf{Sample Percentage}

\[
\hat{p} = \frac{x}{n} \times 100\%
\]

\begin{itemize}
\item $\hat{p}$ - sample proportion.
\item $x$  - number of ``successes".
\item $n$  - the sample size.
\end{itemize}


\begin{itemize}
\item The general structure of confidence intervals is as follows
\[ \mbox{ Point Estimate } \pm \left[ \mbox{ Quantile } \times \mbox{ Standard Error } \right] \]
\item The structure of a confidence interval for sample proportion is
\[ \hat{p} \pm \left[ z_{(\alpha/2)} \times \mbox{S.E.}(\hat{p}) \right]\]

%\item The standard error, in the case of sample proportions, is
%\[ \mbox{S.E.}(\hat{p}) = \sqrt{\frac{\hat{p}\times (1-\hat{p})}{n}}\]

\end{itemize}



\textbf{Point Estimate}:\\
\begin{itemize}
\item The point estimate is the sample proportion, denoted $\hat{p}$.  
\item The sample proportion is calculated as the number of `successes' ($x$) divided by the total number of cases, in other words, the sample size $n$.

\end{itemize}

\noindent \textbf{Quantile}:\\
\begin{itemize}
\item In the cases of large samples ($ n > 30$) , the standard normal ( `Z' ) distribution is used.
\end{itemize}



\noindent \textbf{Standard Error for Proportions}:\\

The standard error for proportions is computed using this formula.
\[
S.E. (\hat{p}) \;=\; \sqrt{ {\hat{p} \times (1-\hat{p} )\over n}}
\]

When expressing the proportion as a percentage, we adjust the standard error accordingly.
\[
S.E. (\hat{p}) \;=\; \sqrt{ {\hat{p} \times (100 -\hat{p} )\over n}}
\]



\subsection{ Sample Proportion : Example}


\begin{description}
\item[Point Estimate] The sample proportion is computed as follows
\[ \hat{p} = \frac{x}{n} = \frac{84}{120} = 0.70 \]
\item[Quantile] We are asked for a 95\% confidence interval. We have a large sample ($n=120$). The quantile is therefore 1.96.
\[ z_{\alpha/2} =1.96\]
\item[Standard Error] The standard error, with sample size n=120 is computed as follows
\[ \mbox{S.E.}(\hat{p}) = \sqrt{\frac{\hat{p} \times (1-\hat{p})}{n}} =  \sqrt{\frac{0.70 \times 0.30}{120}}\]

\end{description}

\textit{Unlike confidence intervals for sample means, there is only one type of confidence interval when dealing with sample proportions.}

%----------------------------------------------------%


\subsection{ Confidence Interval for a proportion}

n = 400

240 


\[\hat{P} = {240 \over 400} = 0.60\]


\[S.E. \;(\hat{P}) = \sqrt{ { \hat{P} \;\times \;(1-\hat{P}) \over n}}= \sqrt{ { 0.60 \;\times \; 0.40 \over 400}}\]



\[S.E. \;(\hat{P}) = \sqrt{ { 60\% \;\times \; 40\% \over 400}}\]














\subsection{Standard Error}

\begin{itemize}
\item The standard error measures the dispersion of the sampling distribution.
\item For each type of point estimate, there is a corresponding standard error.
\item A full list of standard error formulae will be attached in your examination paper.
\item The standard error for a proportion (for confidence intervals only) is
\[ S.E(p)  = \sqrt{{ \hat{p} \times (1- \hat{p}) \over n}} \]
\item
When expressing the proportion as a percentage, we adjust the standard error accordingly.
\[
S.E. (p) \;=\; \sqrt{ {\hat{p} \times (100 -\hat{p} )\over n}}\]
\end{itemize}

\textbf{Standard Error for Proportions}

\begin{itemize}
\item When computing the standard error for computing the confidence intervals, one would use the point estimate $\hat{p}$ in
their calculation.
\item When computing the standard error for computing the hypothesis test, one would use the value for the proportion expected under the null hypothesis $\pi_0$ in
that calculation.
\item For Hypothesis tests The standard error for proportions is computed using this formula.
\[
S.E. (p) \;=\; \sqrt{ {\pi_o \times (1-\pi_o )\over n}}
\]
\end{itemize}



Although the sample mean is useful as an unbiased estimator of the population mean, there is no way of
expressing the degree of accuracy of a point estimator. In fact, mathematically speaking, the probability that the
sample mean is exactly correct as an estimator of the population mean is $P = 0$.



\section{Computing the Standard Error}

\[
S.E. (\hat{p}) \;=\; \sqrt{ {\hat(p) \times (100 -\hat{p} )\over n}}
\]




\[
\hat{p} = {144/200}  \times 100\%  = 0.72 \times 100\%.  = 72%
\]

$100\% - \hat{p} = 100\% - 72\% = 28\% $


\textbf{Computing the Standard Error}

\[
S.E. (\hat{p}) \;=\; \sqrt{ {72 \times 28 \over 200 }}
\]

%==================================================================================================%


%------------------------------------------------------------------------------%
{

\subsubsection{ Standard Error for Proportions}

The standard error for proportions is computed using this formula.
\[
S.E. (\hat{p}) \;=\; \sqrt{ {\hat{p} \times (1-\hat{p} )\over n}}
\]


When expressing the proportion as a percentage, we adjust the standard error accordingly.
\[
S.E. (\hat{p}) \;=\; \sqrt{ {\hat{p} \times (100 -\hat{p} )\over n}}
\]



}






\end{document}
