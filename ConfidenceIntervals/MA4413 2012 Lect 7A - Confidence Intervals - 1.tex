\documentclass[a4]{beamer}
\usepackage{amssymb}
\usepackage{graphicx}
\usepackage{subfigure}
\usepackage{newlfont}
\usepackage{amsmath,amsthm,amsfonts}
%\usepackage{beamerthemesplit}
\usepackage{pgf,pgfarrows,pgfnodes,pgfautomata,pgfheaps,pgfshade}
\usepackage{mathptmx}  % Font Family
\usepackage{helvet}   % Font Family
\usepackage{color}

\mode<presentation> {
 \usetheme{Default} % was
 \useinnertheme{rounded}
 \useoutertheme{infolines}
 \usefonttheme{serif}
 %\usecolortheme{wolverine}
% \usecolortheme{rose}
\usefonttheme{structurebold}
}

\setbeamercovered{dynamic}

\title[MA4704]{Technological Mathematics 4 \\ {\normalsize MA4704 Lecture 6C}}
\author[Kevin O'Brien]{Kevin O'Brien \\ {\scriptsize Kevin.obrien@ul.ie}}
\date{Spring Semester 2013}
\institute[Maths \& Stats]{Dept. of Mathematics \& Statistics, \\ University \textit{of} Limerick}

\renewcommand{\arraystretch}{1.5}

\begin{document}

\begin{frame}
\titlepage
\end{frame}

% - Last lecture
% - symmetric Intervals
% - computing Quantiles
% - student's t distribution
% - CI for means
% - CI for means ( small samples)
% - CI for Props
% - CI for differences.


%-----------------------------------------------------------%


\begin{frame}
\frametitle{Confidence Intervals (Revision) }

\begin{itemize}
\item The $95\%$ confidence interval is a range of values which contain the true population parameter (i.e. mean, proportion etc) with a probability of $95\%$.
\item We can expect that a $95\%$ confidence interval will not include the true parameter values $5\%$ of the time.
\item A confidence level of $95\%$ is commonly used for computing confidence interval, but we could also have confidence levels of $90\%$,$99\%$ and $99.9\%$.
\end{itemize}

\end{frame}

%-----------------------------------------------------------%


\begin{frame}
\frametitle{Confidence Level (Revision) }

\begin{itemize}
\item A confidence level for an interval is denoted to $1-\alpha$ (in percentages: $100(1-\alpha)\%$) for some value $\alpha$.
\item A confidence level of $95\%$ corresponds to $\alpha = 0.05$.
\item $100(1-\alpha)\%$ = $100(1-0.05)\%$  = $100(0.95)\%$ = $95\%$
\item For a confidence level of $99\%$, $\alpha = 0.01$.
\item Knowing the correct value for $\alpha$ is important when determining quantiles.
\end{itemize}

\end{frame}


%-----------------------------------------------------------%
\begin{frame}
\frametitle{The Central Limit Theorem (Revision) }
\begin{itemize}
\item This theorem states that as sample size $n$ is increased, the sampling distribution of the mean (and for other sample statistics as well) approaches the normal distribution in form, regardless of the form of the population distribution from
which the sample was taken.

\item For practical purposes, the sampling distribution of the mean can be assumed to be
approximately normally distributed, even for the most non-normal populations or processes, whenever the
sample size is $n > 30$.

\item (For populations that are only somewhat non-normal, even a smaller sample size will
suffice. A variation of the normal distribution can be used for such circumstances.)
\end{itemize}


\end{frame}
%-----------------------------------------------------------%


\begin{frame}
\frametitle{Computing Confidence Intervals}
Confidence limits are the lower and upper boundaries / values of a confidence interval, that is, the values which define the range of a confidence interval. The general structure of a confidence interval is as follows:

\[ \mbox{Point Estimate}  \pm \left[ \mbox{Quantile} \times \mbox{Standard Error} \right] \]


\begin{itemize}
\item Point Estimate: estimate for population parameter of interest, i.e. sample mean, sample proportion.
\item Quantile: a value from a probability distribution that scales the intervals according to the specified confidence level.
\item Standard Error: measures the dispersion of the sampling distribution for a given sample size $n$.
\end{itemize}
\end{frame}



%-----------------------------------------------------------%

\begin{frame}
\frametitle{Quantiles (1) }

\begin{itemize}
\item The quantile is a value from a probability distribution that scales the intervals according to the specified confidence level.
\item For practical purposes, the quantile can be taken from the standard normal distribution, if the sample is larger than 30, further to the central limit theorem.
\item For a specified confidence level $1-\alpha $, the corresponding quantile is the value $a$ that satisfies the following identity (when $n > 30$):

    \[ p( -a \leq Z \leq a) = 1- \alpha \]

\end{itemize}

\end{frame}

%-----------------------------------------------------------%

\begin{frame}[fragile]
\frametitle{Quantiles (2)}
PCs = Percentiles (i.e. Quantiles)
\begin{verbatim}
>PCs=c(180/200,190:199/200,1998:1999/2000)

>
> PCs
 [1] 0.9000 0.9500 0.9550 0.9600 0.9650 0.9700 0.9750
 [8] 0.9800 0.9850 0.9900 0.9950 0.9990 0.9995
> qnorm(PCs)
 [1] 1.281552 1.644854 1.695398 1.750686 1.811911
 [6] 1.880794 1.959964 2.053749 2.170090 2.326348
[11] 2.575829 3.090232 3.290527
>
\end{verbatim}
\end{frame}
%-----------------------------------------------------------%

\begin{frame}[fragile]
\frametitle{Quantiles (3)}
Negative values for Qauntiles
\begin{verbatim}

>PCs
 [1] 0.1000 0.0500 0.0450 0.0400 0.0350 0.0300 0.0250
 [8] 0.0200 0.0150 0.0100 0.0050 0.0010 0.0005
> qnorm(1-PCs)
 [1] -1.281552 -1.644854 -1.695398 -1.750686 -1.811911
 [6] -1.880794 -1.959964 -2.053749 -2.170090 -2.326348
[11] -2.575829 -3.090232 -3.290527
\end{verbatim}
\end{frame}
%-----------------------------------------------------------%
%-----------------------------------------------------------%

\begin{frame}
\frametitle{Quantiles (4)}

\begin{itemize} \item When the sample size $n$ is greater than 30, we can compute the quantile using Murdoch Barnes table 3, or \texttt{R} code (previous slides).
Remark

\item $95\%$ of Z random variables are between -1.96 ( quantile for $2.5\%$)and 1.96 ( quantile for $97.5\%$)
\end{itemize}

\begin{itemize}
\item If the confidence level is $95\%$, then the quantile is 1.96. Recall
\[ p( -1.96 \leq Z \leq 1.96) = 0.95 \]

\item If the confidence level is $90\%$, then the quantile is 1.645.
\[ p( -1.645 \leq Z \leq 1.645) = 0.90 \]

\item If the confidence level is $99\%$, then the quantile is 2.576.
\[ p( -2.576 \leq Z \leq 2.576) = 0.99 \]

\end{itemize}



\end{frame}


%-----------------------------------------------------------%

\begin{frame}
\frametitle{Standard Error}

\begin{itemize}
\item The standard error measures the dispersion of the sampling distribution.
\item For each type of point estimate, there is a corresponding standard error.
\item A full list of standard error formulae will be attached in your examination paper.
\item The standard error for a  mean is
\[ S.E( \bar{x} )  = {\sigma \over \sqrt{n}} \]
However, we often do not know the value for $\sigma$. For practical purposes, we use the sample standard deviation $s$ as an estimate for $\sigma$ instead.
\[ S.E( \bar{x} )  = {s \over \sqrt{n}} \]
\end{itemize}

\end{frame}


%------------------------------------------------------------------------------%
\frame{

\frametitle{ Standard Error for Proportions}

The standard error for proportions is computed using this formula.
\[
S.E. (\hat{p}) \;=\; \sqrt{ {\hat{p} \times (1-\hat{p} )\over n}}
\]


When expressing the proportion as a percentage, we adjust the standard error accordingly.
\[
S.E. (\hat{p}) \;=\; \sqrt{ {\hat{p} \times (100 -\hat{p} )\over n}}
\]



}
%-----------------------------------------------------------%

\begin{frame}
\frametitle{Margin of Error}

\begin{itemize}
\item The product of the quantile and the standard error give us the width of the confidence interval
\item The width of the confidence interval is known as the \textbf{\emph{margin of error}}.  \[ \mbox{Margin of Error}  = \left[ \mbox{Quantile} \times \mbox{Standard Error} \right] \]
\item The margin of error gives us some idea about how uncertain we are about the unknown population parameter. \item A very wide interval may indicate that more data should be collected before anything very definite can be said about the parameter.
\item The only way to control the margin of error is to adjust the sample size accordingly.
\item By choosing an appropriate sample size, it is possible to ensure that the margin of error does not excess a certain threshold.
\end{itemize}

\end{frame}

