\documentclass[14pt, a4paper]{article}
\usepackage{epsfig}
\usepackage{subfigure}
%\usepackage{amscd}
\usepackage{amssymb}
\usepackage{amsbsy}
\usepackage{amsthm}
%\usepackage[dvips]{graphicx}
\usepackage{natbib}
\bibliographystyle{chicago}
\usepackage{vmargin}
% left top textwidth textheight headheight
% headsep footheight footskip
\setmargins{3.0cm}{2.5cm}{15.5 cm}{22cm}{0.5cm}{0cm}{1cm}{1cm}
\renewcommand{\baselinestretch}{1.5}
\pagenumbering{arabic}
\theoremstyle{plain}
\newtheorem{theorem}{Theorem}[section]
\newtheorem{corollary}[theorem]{Corollary}
\newtheorem{ill}[theorem]{Example}
\newtheorem{lemma}[theorem]{Lemma}
\newtheorem{proposition}[theorem]{Proposition}
\newtheorem{conjecture}[theorem]{Conjecture}
\newtheorem{axiom}{Axiom}
\theoremstyle{definition}
\newtheorem{definition}{Definition}[section]
\newtheorem{notation}{Notation}
\theoremstyle{remark}
\newtheorem{remark}{Remark}[section]
\newtheorem{example}{Example}[section]
\renewcommand{\thenotation}{}
\renewcommand{\thetable}{\thesection.\arabic{table}}
\renewcommand{\thefigure}{\thesection.\arabic{figure}}
\title{MA4605}
\author{ } \date{ }


\begin{document}
\author{Kevin O'Brien}
\title{MA4605}

\tableofcontents \setcounter{tocdepth}{2}


\section{Statistical Inference} What is Statistical Inference?
\begin{itemize}\item Hyptothesis testing \item Confidence
Intervals \item Sample size estimation.\end{itemize}

\subsection{Hypothesis testing: introduction}
The objective of hypothesis testing is to access the validity of a claim against a counterclaim using sample data
\begin{itemize}\item The claim to be “proved” is the alternative hypothesis($H_1$).\item The competing claim is called the null hypothesis($H_0$).\item One begins by assuming that $H_0$ is true. \end{itemize}

If the data fails to contradict $H_0$ beyond a reasonable doubt, then $H_0$ is not rejected. However, failing to reject $H_0$ does not mean that we accept it as true. It simply means that $H_0$ cannot be ruled out as a possible explanation for the observed data. A proof by insufficient data is not a proof at all.

\begin{quote}
“The process by which we use data to answer questions about parameters
is very similar to how juries evaluate evidence about a defendant.” –from
Geoffrey Vining, Statistical Methods for Engineers, Duxbury, 1st edition,
1998.
\end{quote}




\subsection{two populations}

Two samples drawn from two populations are independent samples if
the selection of the sample from population 1 does not affect the
selection of the sample from population 2. The following notation
will be used for the sample and population measurements:

\begin{itemize}
\item $p_1$ and $p_2$ = means of populations 1 and 2,

\item $\sigma_1$ and $\sigma_2$ = standard deviations of
populations 1 and 2,

\item $n_l$ and $n_2$ = sizes of the samples drawn from
populations 1 and 2 ($n_1 >30 $, $n_2 >30 $),

\item $x_1$ and $x_2$, = means of the samples selected from
populations 1 and 2,

\item $s_{1}$ and $s_{2}$ = standard deviations of the samples
selected from populations 1 and 2.

\end{itemize}
\newpage


\subsection{Standard Error}

\begin{equation}
S.E(\bar{X}_{1}-\bar{X}_{2}) =
\sqrt(\frac{s^2_{1}}{n_{1}}+\frac{s^2_{2}}{n_{2}})
\end{equation}

\subsection{Example}
The mean height of adult males is 69 inches and the standard
deviation is 2.5 inches. The mean height of adult females is 65
inches and the standard deviation is 2.5 inches. Let population 1
be the population of male heights, and population 2 the population
of female heights. Suppose samples of 50 each are selected from
both populations.



\subsection{Example 3} Ten replicate analyses of the concentration
of mercury in a sample of commercial gas condensate gave the
following results (in ng/ml) :

\begin{tabular}{|c|c|c|c|c|c|c|c|c|c|}
  \hline
23.3 & 22.5 & 21.9 & 21.5 & 19.9 & 21.3 & 21.7 & 23.8 & 22.6 &
24.7\\
  \hline
\end{tabular}

Compute 99\% confidence limits for the mean.
\section{Hypothesis Tests for Two Means}

If the population standard deviations $\sigma_1$ and $\sigma_2$
are known, the test statistic is of the form:

\begin{equation}
Z = \frac{(\bar{x}_1 - \bar{x}_2) - (\mu_1 - \mu_2 ) }{\sqrt{
\frac{\sigma^2_1}{n_1}+\frac{\sigma^2_2}{n_2}} }
\end{equation}
The critical value and p-value are looked up in the normal tables.






\subsection{F-test of equality of variances}
The test statistic is

\begin{equation} F = \frac{S_X^2}{S_Y^2}\end{equation}

has an F-distribution with $n-1$ and $m-1$ degrees of freedom if the null hypothesis of equality of variances is true.


\chapter{Further Inference}

\section{Fractional factorial design}

(d)	Define the following terms used in fractional factorial design; Defining relation,
	Generator, Confounding, Resolution. Which design resolution is considered
	optimal?




