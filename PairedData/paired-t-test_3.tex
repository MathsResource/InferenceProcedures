\documentclass[a4]{beamer}
\usepackage{amssymb}
\usepackage{graphicx}
\usepackage{subfigure}
\usepackage{framed}
\usepackage{newlfont}
\usepackage{amsmath,amsthm,amsfonts}
%\usepackage{beamerthemesplit}
\usepackage{pgf,pgfarrows,pgfnodes,pgfautomata,pgfheaps,pgfshade}
\usepackage{mathptmx}  % Font Family
\usepackage{helvet}   % Font Family
\usepackage{color}

\mode<presentation> {
 \usetheme{Frankfurt} % was
 \useinnertheme{rounded}
 \useoutertheme{infolines}
 \usefonttheme{serif}
 %\usecolortheme{wolverine}
% \usecolortheme{rose}
\usefonttheme{structurebold}
}

\setbeamercovered{dynamic}

\title[MA4603]{Science Maths 3 \\ {\normalsize MA4603 Lecture 11A}}
\author[Kevin O'Brien]{Kevin O'Brien \\ {\scriptsize Kevin.obrien@ul.ie}}
\date{Autumn Semester 2017}
\institute[Maths \& Stats]{Dept. of Mathematics \& Statistics, \\ University \textit{of} Limerick}

\renewcommand{\arraystretch}{1.5}

\begin{document}

%------------------------------------------------------------------------%
\frame{
\frametitle{Paired T test}
\large
\begin{itemize}
\item Firstly we have to compute each of the case-wise differences.
\item Then we have to compute the mean value of these differences.
\item Lastly we also have to compute the standard deviation of the differences.
\end{itemize}
}
%----------------------------------------------------------------------%
\frame{
To test the null hypothesis that the true mean difference is zero, the procedure is as
follows:
1. Calculate the difference $(di = y_i − x_i)$ between the two observations on each pair,
making sure you distinguish between positive and negative differences.
2. Calculate the mean difference, $\bar{d}$.


3. Calculate the standard deviation of the differences, $s_d$, and use this to calculate the
standard error of the mean difference, $SE(\bar{d}) = {s_d \over \sqrt{n}}$

4. Calculate the t-statistic, which is given by $ T ={ \bar{d} \over SE(\bar{d})}$.

Under the null hypothesis, this statistic follows a t-distribution with $n − 1$ degrees of freedom.

}

\end{document}