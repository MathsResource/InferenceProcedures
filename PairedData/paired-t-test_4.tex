
\frame{
\frametitle{The Paired t-test}
A paired t-test is used to compare two population means where you have two samples in
which observations in one sample can be paired with observations in the other sample.\\
\bigskip
Examples of where this might occur are:
\begin{itemize}
\item Before-and-after observations on the same subjects (e.g. students� diagnostic test
results before and after a particular module or course).
\item A comparison of two different methods of measurement or two different treatments
where the measurements/treatments are applied to the \textbf{\emph{same}} subjects.
\end{itemize}
}



%-------------------------------------------------------------------------------------------%
\begin{frame}
\frametitle{The Paired t-test}
\begin{itemize}
\item We will often be required to compute the case-wise differences, the average of those differences and the standard deviation of those difference.

\item The mean difference for a set of differences between paired observations is
\[ \bar{d} = \sum d_i \over n \]

\item The computational formula for the standard deviation of the differences
between paired observations is
\[s_d = \sqrt{ {\sum d_i^2 - n\bar{d}^2 \over n-1}}\]
\item It is nearly always a small sample test.
\end{itemize}
\end{frame}


%----------------------------------------------------------------------------------------------------%
\frame{
\frametitle{Paired T test}
\begin{itemize}
\item $\mu_d$ mean value for the population of differences.
\item The null hypothesis is that that $\mu_d = 0$
\item Given $\bar{d}$ mean value for the sample of differences, and $s_d$ standard deviation of the differences for the paired sample data, we can compute this test in the same manner as a one-sample test for the mean
\end{itemize}
}



