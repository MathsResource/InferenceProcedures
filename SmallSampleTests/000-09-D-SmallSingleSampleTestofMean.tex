s\documentclass[]{report}

\voffset=-1.5cm
\oddsidemargin=0.0cm
\textwidth = 480pt

\usepackage{framed}
\usepackage{subfiles}
\usepackage{enumerate}
\usepackage{graphics}
\usepackage{newlfont}
\usepackage{eurosym}
\usepackage{amsmath,amsthm,amsfonts}
\usepackage{amsmath}
\usepackage{color}
\usepackage{amssymb}
\usepackage{multicol}
\usepackage[dvipsnames]{xcolor}
\usepackage{graphicx}
\begin{document}


\section{Hypothesis test for the mean of a single sample}

This procedure is used to assess whether the population mean  has a specified value, based on the sample mean. The hypotheses are conventionally written in a form similar to below (here the hypothesized population mean is zero).



There are two hypothesis test for the mean of a single sample.

\begin{enumerate}
\item The sample is of a normally-distributed variable for which the population standard deviation ($\sigma$) is known. 
\item The sample is of a normally-distributed variable where σ is estimated by the sample standard deviation (s).
\end{enumerate} 
In practice, the population standard deviation is rarely known. For this reason, we will consider the second case only in this course.

In most statistical packages, this analysis is performed in the summary statistics functions.




\subsection*{The null and alternative hypotheses}
\[ H_0: \mu = 40 kg\]
\[ H_1: \mu \neq 40 kg\]


\section{Independent one-sample $t$-test}
In testing the null hypothesis that the population mean is equal to a specified value $\mu_{0}$, one uses the statistic

\begin{equation}t = \frac{\overline{x} - \mu_0}{s / \sqrt{n}}\end{equation}

where $s$ is the sample standard deviation and $n$ is the sample size. The degrees of freedom used in this test is $n - 1$.




\section{Inference Procedure: Worked Examples}



The mean and variance of height in a sample of 25 Irish students are 174cm and 100cm2, respectively.

Test the hypothesis that the mean height of all Irish students is 170cm at a significance level of 5<%. 
%%%%%%%%%%%%%%%%%%%%%%%%%%%%%%%%%%%%%%%%%%%%%%%%%%%%%%%%%%%%%%%%%%%%%%%%%%%%%%%%%%%%%%%%%%%%%%%%%%%%%%%


\section*{Part 1: One Sample t-test}
Question here

\subsection*{The null and alternative hypotheses}
\[ H_0: \mu = 40 kg\]
\[ H_1: \mu \neq 40 kg\]




\subsection{2 sided test}
A two-sided test is used when we are concerned about a possible
deviation in either direction from the hypothesized value of the
mean. The formula used to establish the critical values of the
sample mean is similar to the formula for determining confidence
limits for estimating the population mean, except that the
hypothesized value of the population mean $\mu_0$ is the reference
point rather than the sample mean.







\section{Hypothesis Tests for single samples}

\begin{itemize}
\item We could have inference procedures for single sample studies. We would base an argument on the either the sample mean or sample proportion as appropriate.
\item A hypothesis test can be used to determine how ``confident" we can be with our data in making that statements.
\item The lower the significance level (The margin for Type I error) the stronger our data must be.
\item Large samples lead to more confident conclusion.

\medskip

\item We could have either hypothesis test for the sample mean or the sample proportion, to test a statement about the population as a whole (i.e something about the population mean)
\item We make our argument in the form of the null and alternative hypotheses. 
\item The Hypothesis testing procedure determines the strength of evidence in making our arguments. 

\medskip

\item We simply follow the four step procedure. 
\item All of the components are the same used in confidence intervals.
\item The critical value is simply a quantile from the $Z$ or $t-$distribution.
\item The standard errors are also as before. Although when performing a hypothesis test for proportions, we use the expected value under the null hypothesis, rather than point estimate. (reason beyond scope of course.)
\end{itemize}


%--------------------------------------------------------------------------------------%






\end{document}



