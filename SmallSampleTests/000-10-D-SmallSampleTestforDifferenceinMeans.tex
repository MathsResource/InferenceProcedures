
\documentclass[]{report}

\voffset=-1.5cm
\oddsidemargin=0.0cm
\textwidth = 480pt

\usepackage{framed}
\usepackage{subfiles}
\usepackage{graphics}
\usepackage{newlfont}
\usepackage{eurosym}
\usepackage{amsmath,amsthm,amsfonts}
\usepackage{amsmath}
\usepackage{color}
\usepackage{amssymb}
\usepackage{multicol}
\usepackage[dvipsnames]{xcolor}
\usepackage{graphicx}
\begin{document}
	
	%---------------------------------------------------------%
	
	\section{Difference Of Two Means}
	%-http://onlinestatbook.com/chapter8/difference_means.html
	In order to construct a confidence interval, we are going to make three assumptions:
	
	\begin{itemize}
		\item The two populations have the same variance. This assumption is called the assumption of homogeneity of variance.
		\item For the time being, we will use this assumption. Later on in the course, we will discuss the validity of this assumption for two given samples.
		\item The populations are normally distributed.
		\item Each value is sampled independently from each other value.
	\end{itemize}
	
	
	%---------------------------------------------------------%
	
	\section{Computing the Confidence Interval}
	
	\begin{itemize}
		\item As always the first step is to compute the point estimate. For the difference of means for groups $X$ and $Y$, the point estimate is simply the difference between the two means i.e. $\bar{x} - \bar{y}$.
		
		\item As we have seen previously, sample size has a bearing in computing both the quantile and the standard error.
		For two groups, we will use the aggregate sample size ($n_x+n_y)$ to compute the quantile. (For the time being we will assume, the aggregate sample size is large ($n_x+n_y)> 30$.)
		
		\item Lastly we must compute the standard error $S.E.(\bar{x}-\bar{y})$. The formula for computing standard error for the difference of two means, depends on whether or not the aggregate sample size is large or not. For the case that the sample size is large, we use the following formula (next slide).
	\end{itemize}
	
	
	%---------------------------------------------------------%
	
	\section{Computing the Confidence Interval}
	Standard Error for difference of two means (large sample)
	
	\[ S.E.(\bar{x}-\bar{y}) = \sqrt{\frac{s^2_x}{n_x} + \frac{s^2_y}{n_y}} \]
	
	\begin{itemize}
		\item $s^2_x$ and $s^2_x$ is the variance of samples $X$ and $Y$ respectively.
		\item $n_x$ and $n_y$ is the sample size of both samples.\bigskip
		
		\item For small samples, the degrees of freedom is $df = n_x + n_y - 2$. If the sample size $n \leq 32$, we can find appropriate $t-$quantile, rather than assuming it is a $z-$quantile.
	\end{itemize}
	
	
	%---------------------------------------------------------%
	
	\section{CI for Difference in Two Means}
	A research company is comparing computers from two different companies, X-Cel and Yellow, on the basis of energy consumption per hour. Given the following data, compute a $95\%$ confidence interval for the difference in energy consumption.
	\begin{center}
		\begin{tabular}{|c|c|c|c|}
			\hline
			Type & sample size & mean & variance \\ \hline
			X-cel & 17 & 5.353 & 2.743 \\ \hline
			Yellow & 17 & 3.882 & 2.985 \\ \hline
		\end{tabular}
	\end{center}
	Remark: It is reasonable to believe that the variances of both groups is the same. Be mindful of this.
	
	
	%---------------------------------------------------------%
	
	\begin{itemize}
		\item Point estimate : $\bar{x} - \bar{y}$ = 1.469
		\item Standard Error: 0.5805
		\[ S.E.(\bar{x}-\bar{y}) = \sqrt{\frac{2.743}{17} + \frac{2.985}{17}} = \sqrt{0.33698} \]
		\item Quantile : 1.96 (Large sample, with confidence level of $95\%$.)
	\end{itemize}
	
	\[ 1.469  \pm (1.96 \times 0.5805) = (0.3321,2.607) \]
	
	
	This analysis provides evidence that the mean consumption level per hour for X-cel is higher than the mean consumption level per hour for Yellow, and that the difference between means in the population is likely to be between 0.332 and 2.607 units.
	
	
	%---------------------------------------------------------%
	
	
	\section{Computing the Confidence Interval}
	Standard Error for difference of two means (small aggregate sample)
	
	\[ S.E.(\bar{x}-\bar{y}) = \sqrt{  s^2_p \left({1\over n_x}+{1\over n_y} \right)} \]
	
	Pooled Variance $s^2_p$ is computed as:
	
	\[ s^2_p = \frac{(n_x-1)s^2_x + (n_y-1)s^2_y}{(n_x-1) + (n_y-1)} \]
	
	%---------------------------------------------------------%
	
	\section{CI for Difference in Two Means}
	From the previous example (comparing X-cel and Yellow) lets compute a 95\% confidence interval when the sample sizes are $n_x=10$ and $n_y=12$ respectively. (Lets assume the other values remain as they are.)
	\begin{center}
		\begin{tabular}{|c|c|c|c|}
			\hline
			Type & sample size & mean & variance \\ \hline
			X-cel & 10 & 5.353 & 2.743 \\ \hline
			Yellow & 12 & 3.882 & 2.985 \\ \hline
		\end{tabular}
	\end{center}
	The point estimate $\bar{x} - \bar{y}$ remains as 1.469. Also we require that both samples have equal variance. As both $X$ and $Y$ have variances at a similar level, we will assume equal variance.
	
	
	%---------------------------------------------------------%
	
	
	\section{Computing the Confidence Interval}
	\begin{itemize} \item Pooled variance $s^2_p$ is computed as:
		
		\[ s^2_p = \frac{(10-1)2.743 + (12-1)2.985}{(10-1) + (12-1)}  = \frac{57.52}{20} = 2.87\]
		
		\item Standard error for difference of two means is therefore
		
		\[ S.E.(\bar{x}-\bar{y}) = \sqrt{  2.87 \left({1\over 10}+{1\over 12} \right)} = 0.726 \]
		
		\item The aggregate sample size is small i.e. 22. The degrees of freedom is $n_x+n_y-2 = 20$.
		From Murdoch Barnes tables 7, the quantile for a $95\%$ confidence interval is 2.086.
		
		\item The confidence interval is therefore
		\[ 1.469  \pm (2.086 \times 0.726) = 1.4699 \pm 1.514 =  (-0.044, 2.984 )  \]
	\end{itemize}
	
	
	

	
	\subsection{Difference in Two means}
	For this calculation, we will assume that the variances in each of the two populations are equal. This assumption is called the assumption of homogeneity of variance.
	
	The first step is to compute the estimate of the standard error of the difference between means ().
	
	\[ S.E.(\bar{X}-\bar{Y}) = \sqrt{\frac{s^2_x}{n_x} + \frac{s^2_y}{n_y}} \]
	
	\begin{itemize}
		\item $s^2_x$ and $s^2_x$ is the variance of both samples.
		\item $n_x$ and $n_y$ is the sample size of both samples.
	\end{itemize}
	The degrees of freedom is $n_x + n_y -2$.
	
	
	
	%----------------------------------------------------% 
	
	\subsection{Confidence Interval for a Mean (Small Sample)}
	\begin{itemize}
		\item The mean operating life for a random sample of $n = 10$ light bulbs is $\bar{x} = 4,000$ hours, with the sample
		standard deviation $s = 200$ hours. \item The operating life of bulbs in general is assumed to be approximately normally distributed.\item
		We estimate the mean operating life for the population of bulbs from which this sample was taken, using a 95 percent
		confidence interval as follows:
		
		\[4,000\pm(2.262)(63.3)  = (3857,4143)\]
		
		\item The point estimate is 4,000 hours. The sample standard deviation is 200 hours, and the sample size is 10. Hence \[S.E(\bar{x} ) = { 200 \over \sqrt{10}} = 63.3\]
		
		\item From last slide, the t quantile with $df=9$ is 2.262.
	\end{itemize}
	
	
\end{document}