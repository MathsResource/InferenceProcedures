	\documentclass[a4paper,12pt]{article}
%%%%%%%%%%%%%%%%%%%%%%%%%%%%%%%%%%%%%%%%%%%%%%%%%%%%%%%%%%%%%%%%%%%%%%%%%%%%%%%%%%%%%%%%%%%%%%%%%%%%%%%%%%%%%%%%%%%%%%%%%%%%%%%%%%%%%%%%%%%%%%%%%%%%%%%%%%%%%%%%%%%%%%%%%%%%%%%%%%%%%%%%%%%%%%%%%%%%%%%%%%%%%%%%%%%%%%%%%%%%%%%%%%%%%%%%%%%%%%%%%%%%%%%%%%%%
\usepackage{eurosym}
\usepackage{vmargin}
\usepackage{amsmath}
\usepackage{framed}
\usepackage{graphics}
\usepackage{epsfig}
\usepackage{subfigure}
\usepackage{enumerate}
\usepackage{fancyhdr}

\setcounter{MaxMatrixCols}{10}
%TCIDATA{OutputFilter=LATEX.DLL}
%TCIDATA{Version=5.00.0.2570}
%TCIDATA{<META NAME="SaveForMode"CONTENT="1">}
%TCIDATA{LastRevised=Wednesday, February 23, 201113:24:34}
%TCIDATA{<META NAME="GraphicsSave" CONTENT="32">}
%TCIDATA{Language=American English}

\pagestyle{fancy}
\setmarginsrb{20mm}{0mm}{20mm}{25mm}{12mm}{11mm}{0mm}{11mm}
\lhead{Statistics with \texttt{R}} \rhead{Kevin O'Brien} \chead{Proportions Test} %\input{tcilatex}

\begin{document}


\subsection{Hypothesis test of Proportion}
This procedure is used to assess whether an assumed proportion is supported by evidence. For two tailed tests, the null hypothesis states that the population proportion  π has a specified value, with the alternative stating that π has a different value. 

The hypotheses are typically as follows:   

\begin{itemize}
	\item[Ho] : $\pi = 0.50$
	\item[Ha] : $\pi \neq 0.50$
\end{itemize}

\subsubsection{Example}
\begin{itemize}
    \item A manufacturer is interested in whether people can tell the difference between a new formulation of a soft drink and the original formulation. The new formulation is cheaper to produce so if people cannot tell the difference, the new formulation will be manufactured. 

    \item A sample of 100 people is taken. Each person is given a taste of both formulations and asked to identify the original. Sixty-two percent of the subjects correctly identified the new formulation. Is this proportion significantly different from $50\%$? 

    \item The first step in hypothesis testing is to specify the null hypothesis and an alternative hypothesis. In testing proportions, the null hypothesis is that $\pi$, the proportion in the population, is equal to 0.5. The alternate hypothesis is $\pi \neq 0.5$. 

    \item The computed p-values is compared to the pre-specified significance level of $5\%$. Since the p-value (0.0214) is less than the significance level of 0.05, the effect is statistically significant. 
\end{itemize}

\begin{framed}
\begin{verbatim}
> prop.test(62,100,0.5)

1-sample proportions test with continuity correction

data:  62 out of 100, null probability 0.5 
X-squared = 5.29, df = 1, p-value = 0.02145
alternative hypothesis: true p is not equal to 0.5 
95 percent confidence interval:
0.5170589 0.7136053 
sample estimates:
p 
0.62 
\end{verbatim}
\end{framed}
Since the effect is significant, the null hypothesis is rejected. It is concluded that the proportion of people choosing the original formulation is greater than 0.50. 

This result might be described in a report as follows: 

\begin{quote}
	The proportion of subjects choosing the original formulation (0.62) was significantly greater than 0.50, with p-value = 0.021.
\end{quote}  



\newpage
\subsubsection{Example}
A manufacturer is interested in whether people can tell the difference between a new formulation of a soft drink and the original formulation. The new formulation is cheaper to produce so if people cannot tell the difference, the new formulation will be manufactured. 

A sample of 100 people is taken. Each person is given a taste of both formulations and asked to identify the original. Sixty-two percent of the subjects correctly identified the new formulation. Is this proportion significantly different from $50\%$? 

The first step in hypothesis testing is to specify the null hypothesis and an alternative hypothesis. In testing proportions, the null hypothesis is that $\pi$, the proportion in the population, is equal to 0.5. The alternate hypothesis is $\pi \neq 0.5$. 

The computed p-values is compared to the pre-specified significance level of $5\%$. Since the p-value (0.0214) is less than the significance level of 0.05, the effect is statistically significant. 
\begin{framed}
\begin{verbatim}
> prop.test(62,100,0.5)

1-sample proportions test with continuity correction

data:  62 out of 100, null probability 0.5 
X-squared = 5.29, df = 1, p-value = 0.02145
alternative hypothesis: true p is not equal to 0.5 
95 percent confidence interval:
0.5170589 0.7136053 
sample estimates:
p 
0.62 
\end{verbatim}
\end{framed}
Since the effect is significant, the null hypothesis is rejected. It is concluded that the proportion of people choosing the original formulation is greater than 0.50. 

This result might be described in a report as follows: 

\begin{quote}
	The proportion of subjects choosing the original formulation (0.62) was significantly greater than 0.50, with p-value = 0.021.
\end{quote}  

%----------------------------------------------------%
\end{document}