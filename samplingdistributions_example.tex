
\subsection{sampling distributions}
Suppose that the heights of students are normally distributed with a
mean of 68.5 inches and a standard deviation of 2.7 inches.\\ If 200 random samples of
size 25 are drawn from this population and their means recorded to the nearest tenth
of an inch, determine:
\begin{itemize}
\item[(1)] The expected mean and standard deviation of the sampling distribution of the
mean.
\item[(2)] The expected number of recorded sample means that fall between 67.9 and 69.2
inclusive.
\item[(3)] The expected number of recorded sample means falling below 67.0.
\end{itemize}

%-------------------------------------------------------------------------------%


\subsection{Sampling distributions}

The sampling distribution of the mean of 25 observations has the same mean
as the population, which is 68.5 inches, and standard deviation from (5.3) of
2.7/5 = 0.54.\\ \bigskip 

The samples are random, so one can’t be sure just how many will have means
recorded between 67.9 and 69.2 inches. One can work out the probability that
a recorded mean will so lie, from the sampling distribution of the sample mean
which is normal with mean 68.5 and standard deviation 0.54. \\ \bigskip 

That probability is the probability that the sample mean lies between 67.85 and 69.25, allowing
for the crude recording of the means. The corresponding z scores are
(67.85 − 68.5)/0.54 = −1.20 and (69.25 − 68.5)/0.54 = 1.39.

%-------------------------------------------------------------------------------%

\subsection{Sampling distributions}
The probability between the two z values is
\[0.9177 − 0.1151 = 0.8026.\]
Since there are 200 samples drawn, you can now think of each as a single trial.
The recorded mean lies between 67.9 and 69.2 with probability 0.8026 at each
trial.

We are dealing with a binomial distribution with n = 200 trials and
probability of success $p = 0.8026$. The expected number of successes is
\[np = 200 × 0.8026 = 160.52.\]

%-------------------------------------------------------------------------------%

Similarly, the probability that a recorded mean lies below 67.0 is the probability
that the sample mean lies below 66.95. The z value is (66.95 − 68.5)/0.54 =
−2.87. We want the left-hand tail probability for z = −2.87, which is the
right-hand tail probability for z = 2.87. \\  \bigskip { From Table 4 of the New Cambridge
}Statistical Tables this is 0.00205. So the expected number of sample means out
of 200 recorded below 67.0 is $200 \times 0.00205 = 0.41$.


%-------------------------------------------------------------------------------------------------------------------%
{

The mean of the sampling distribution is the mean of the population

$\mu_{\bar{x}}$ = $\mu$

The standard deviation of the sampling distribution is the square root of the standard deviation of the population.


We call this standard deviation the \textbf{\emph{Standard Error}} .

\[ \sigma_{\bar{x}} = {\sigma \over\sqrt{ n} }\]
}

