\documentclass[]{report}

\voffset=-1.5cm
\oddsidemargin=0.0cm
\textwidth = 480pt

\usepackage{framed}
\usepackage{subfiles}
\usepackage{graphics}
\usepackage{newlfont}
\usepackage{eurosym}
\usepackage{enumerate}
\usepackage{amsmath,amsthm,amsfonts}
\usepackage{amsmath}
\usepackage{color}
\usepackage{amssymb}
\usepackage{multicol}
\usepackage[dvipsnames]{xcolor}
\usepackage{graphicx}
\begin{document}
%====================================================================================================%


MA4413 Tutorials for Week 9




\subsection*{Question 10} 
A survey of study habits wishes to determine whether the meanstudy hours completed by women at a particular college is higher
than for men at the same college. A sample of $n_1$ = 10 women and
$n_2$ = 12 men were taken, with mean hours of study $\bar{x}_1$ =
120 and $\bar{x}_2$ = 105 respectively. The standard deviations
were known to be $\sigma_1$ = 28 and $\sigma_2$ = 35.

The hypothesis being tested is:

\begin{eqnarray}
H_{0}: \mu_1 = \mu_2\qquad \qquad (\mu_1 - \mu_2= 0)\\
H_{a}: \mu_1 \neq \mu_2 \qquad \qquad (\mu_1 - \mu_2 \neq 0)
\end{eqnarray}

In $R$, the test statistic is calculated using:

\begin{verbatim}
xbar1 <- 120
xbar2 <- 105
sd1 <- 28
sd2 <- 35
n1 <- 10
n2 <-12
TS <- ( (xbar1 - xbar2) - (0) )/sqrt( (sd1^2/n1) + (sd2^2/n2) )
TS
[1] 1.116536
\end{verbatim}
Now need to calculate the critical value or the p-value.


The critical value can be looked up using qnorm. Since this is a
one-tailed test and there is a > sign in $H_1$:

\begin{verbatim}
qnorm(0.95)
[1] 1.644854
\end{verbatim}

Since the test statistic is less than the critical value ( 1.116536 < 1:645 )there is not enough evidence to reject $H_0$
and conclude that the population mean hours study for women is
not higher than the population mean hours study for men.


The p-value is determined using pnorm.

Careful! Remember pnorm
gives the probability of getting a value LESS than the value specified. We want the probability of getting a value greater than
the test statistic.
\begin{verbatim}
1-pnorm(1.116536) # OR pnorm(1.116536, lower.tail=FALSE)
[1] 0.1320964
\end{verbatim}
\subsection*{Question 8} 

\begin{itemize}
\item 
A survey, carried out at a major flower and gardening show, was concerned with the association between the intention to return to the show next year and the purchase of goods at this year s show.
\item There were 220 people interviewed and of these 101 had made a purchase; 69 of these people said they intended to return next year. 
\item Of the 119 who had not made a purchase, 68 said they intended to return next year.
\item By testing the difference between the proportions of purchasers and non-purchasers who intend to return next year, examine whether there is sufficient evidence to justify concluding that the intention to return depends on whether or not a purchase was made.
\end{itemize}

\begin{description}
\item[H0]: population proportions of those who intend to return are equal
\item[H1]: population proportions of those who intend to return are NOT equal
\end{description}

\begin{itemize}
\item Proportion of purchasers 1 :  69 /101; 
\item proportion of non-purchasers 2 : 68 /119
\end{itemize}


Observed value of D = 0.1117


Estimated standard error of D = 6.558\%

\end{document}
