\section{Introduction to Inference}
\begin{itemize}
\item This chapter is concerned with data based decision-making. It is about making a
decision which involves a population. The population is made up of a set of
individual items. 
\item This could be, for example, a set of individuals or companies which
constitute the market for your product. It could consist of the items being
manufactured from a production line.\\
\item The sort of information needed for a decision may be a mean value, (e.g. How many
items does an individual purchase per year, on average?) or a proportion (What
proportion of items manufactured have a fault?). The associated decision may range
from setting up extra capacity to cope with estimated demand, to stopping the
production line for readjustment.\\
\item In most cases it is impossible to gather information about the whole population, so
one has to collect information about a sample from the population and infer the
required information about the population.\\
\end{itemize}


