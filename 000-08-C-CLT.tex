\documentclass[]{report}

\voffset=-1.5cm
\oddsidemargin=0.0cm
\textwidth = 480pt

\usepackage{framed}
\usepackage{subfiles}
\usepackage{graphics}
\usepackage{newlfont}
\usepackage{eurosym}
\usepackage{amsmath,amsthm,amsfonts}
\usepackage{amsmath}
\usepackage{color}
\usepackage{amssymb}
\usepackage{multicol}
\usepackage[dvipsnames]{xcolor}
\usepackage{graphicx}
\begin{document}
%---------------------------------------------------------------%


\noindent \textbf{Sampling Distribution}
\begin{itemize}

\item A probability distribution of a statistic obtained through a large number of samples drawn from a specific population. 
\item The sampling distribution of a given population is the distribution of frequencies of a range of different outcomes 
that could possibly occur for a statistic of a population. 
\end{itemize}

\section{The Central Limit Theorem}
The central limit theorem allows statisticians to use sample statistics to make inferences about the population parameters without knowing about the distribution of the parent population .




\begin{itemize}
\item The main aspect of the CLT that we shall consider is that many statistics (e.g sample mean and other related statistics, such as the sample variance) are normally distributed.
\item Consider the population, characterized by the histogram on the next slide.
\item While the population is not normally distributed, the population of sample statistic will be normally distributed.
\end{itemize}

%------------------------------------------%

\textbf{Central Limit Theorem}
\begin{figure}
% Requires \usepackage{graphicx}
\includegraphics[scale=0.30]{images/6BPopHist}\\

\end{figure}


%------------------------------------------%

\textbf{Central Limit Theorem}
\begin{itemize}
\item Consider an experiment whereby a sample of 60 members of this population was taken, and the following sample statistics were computed
\begin{itemize}
\item Sample mean $\bar{X}$
\item Sample variance $s^2$
\item Sample median $\tilde{X}$
\end{itemize}
\item This experiment was performed 100 times ( i.e. 100 independent samples were taken).
\item The sample statistics were collated by type and examined to determine normality.
\item A data set can be tested for normality using a very simple graphical procedure known as the `Normal Probability Plot', or Q-Q plot.
\item A data set can be assumed to be normally distributed if the points on the Q-Q plot follow the trendline.
\end{itemize}

%------------------------------------------%
%------------------------------------------%

\textbf{CLT: Sample Mean Q-Q plot}
\begin{figure}
% Requires \usepackage{graphicx}
\includegraphics[scale=0.30]{images/6BmeanQQplot}\\

\end{figure}


%------------------------------------------%

\textbf{CLT: Sample Variance Q-Q plot}
\begin{figure}
% Requires \usepackage{graphicx}
\includegraphics[scale=0.30]{images/6BvarQQplot}\\

\end{figure}


\begin{figure}
% Requires \usepackage{graphicx}
\includegraphics[scale=0.30]{images/6BmedianQQplot}\\

\end{figure}

%------------------------------------------%

\textbf{Central Limit Theorem: Sampling Distributions}
\begin{itemize}
\item In each of the three plots, the points follow the trend-line quite closely in each case.
\item As we can see, the population of these statistics are normally distributed.
\item We refer to these distributions as `Sampling Distributions'.
\item While the statistic that we will be dealing with in this module do have normal sampling distributions, it must be noted that many statistics have sampling distributions other than the normal distribution.
\end{itemize}

%------------------------------------------%

\textbf{Quantile Functions}
\begin{itemize} \item The Cumulative Distribution Function is used to identify the probability of a random variable being below a threshold value $k$.
\[P(X \leq k) \]

\item In short, we compute a probability values associated with a specified value.
\item (In \texttt{R}, this is carried out using the \texttt{p-} family of functions.)
\item With Quantile Functions, we are performing the opposite operation, i.e. for a specified probability, we determine the threshold value $k$.
\item For some value $p$, we computed $k$ such that
\[P(X \leq k) = p\]
\item (In \texttt{R}, this is performed using the \texttt{q-} family of functions.)
\end{itemize}




%------------------------------------------%

\textbf{Quantile Functions}
\begin{itemize}
\item Recall that, from the Murdoch Barnes Tables, $P(Z \leq 1.96) = 0.975$ and $P(Z \leq 1.28) = 0.8997$
\end{itemize}
\begin{verbatim}
> qnorm(0.975)
[1] 1.959964
>
> qnorm(0.8997)
[1] 1.279844
\end{verbatim}

\begin{verbatim}
Z = c(rnorm(300,10,3) , rnorm(150,15,1) , rnorm(100,24,3.5),rnorm(200,30,4) , rnorm(400,45,6),rnorm(500,60,2))
Population =Z
hist(Population, breaks = -1:69, col=c("midnightblue","lightblue","slateblue"))



Var.Sample = numeric()
Median.Sample = numeric()
Mean.Sample = numeric()

for( i in 1:100)
{
Sample = sample(Z,60)


Mean.Sample = c(Mean.Sample,mean(Sample))
Median.Sample = c(Median.Sample,median(Sample))
Var.Sample = c(Var.Sample,var(Sample))

}
qqnorm(Median.Sample, pch =18, col="red")
qqline(Median.Sample)
#
qqnorm(Mean.Sample, pch =18, col="red")
qqline(Mean.Sample)
#



qqnorm(Var.Sample, pch =18, col="red")
qqline(Var.Sample)

qqnorm(Median.Sample, pch =18, col="red")
qqline(Median.Sample)
\end{verbatim}

\section{The Central Limit Theorem }
\begin{itemize}
\item This theorem states that as sample size $n$ is increased, the sampling distribution of the mean (and for other sample statistics as well) approaches the normal distribution in form, regardless of the form of the population distribution from
which the sample was taken.
\item For practical purposes, the sampling distribution of the mean can be assumed to be
approximately normally distributed, even for the most non-normal populations or processes, whenever the
sample size is $n > 30$.
\item (For populations that are only somewhat non-normal, even a smaller sample size will
suffice. A variation of the normal distribution can be used for such circumstances.)
\end{itemize}




\section{Central Limit Theorem}

\begin{itemize}

\item Before we can begin computing confidence intervals, we must introduce the \textbf{\emph{Central Limit Theorem}}.

\item Suppose random sample of size $n$ are drawn from any distribution, with the distribution having a mean of $\mu$ (equivalently $E(X)$) and variance of $\sigma^2$ (i.e. standard deviation of $\sigma$).

\item Also suppose that the sample size is large ( i.e. $n > 30$ ).

\item The sample means tend to form a normal distribution with mean $\mu$ and standard deviation $ { \sigma \over \sqrt{n} }$

\item We call the standard deviation of the sample means the \textbf{\emph{standard error}}
\item Standard error is commonly denoted as $S.E.$
\end{itemize}


\section{The Central Limit Theorem}
The central limit theorem allows statisticians to use sample statistics to make inferences about the population parameters without knowing about the distribution of the parent population .



\end{document}
